\documentclass{article}
\usepackage[a4paper,top=1cm,left=1cm,right=1cm,bottom=2cm]{geometry}
\usepackage{bm}
\usepackage{amsmath}
\usepackage{amssymb}
\usepackage{algorithm}
\usepackage{algorithmic}
\usepackage{tikz}
\usetikzlibrary{positioning,shapes,shadows,arrows,matrix}

\title{Nektar++: Incorporating iterative solvers}
\author{C. D. Cantwell}
\date{\today}

\begin{document}
\maketitle
\tikzstyle{class}=[rectangle, draw=black, rounded corners, fill=blue!40, drop
shadow, text centered, anchor=north, text=white, minimum width=3cm,
rectangle split, rectangle split parts=3]
\tikzstyle{routine}=[rectangle, draw=black, rounded corners, fill=green, drop
shadow, text centered, anchor=north, text=white, text width=5cm, node
distance=1cm] 
\tikzstyle{proposed}=[rectangle, draw=black, rounded corners,
fill=red!50, drop shadow, text centered, anchor=north, text=white, text
width=5cm, node distance=1cm] 
\tikzstyle{myarrow}=[->, >=open triangle 90, thick]
\tikzstyle{line}=[-, thick]


\section{Specification}
Implement the necessary framework of classes and interfaces to:
\begin{itemize}
  \item Implement the conjugate gradient method for solving global systems
  iteratively.
  \item Provide a unified interface for linking to other iterative
  and/or parallel iterative solver libraries.
  \item Encapsulate the underlying interaction with external libraries into a
  common interface shared across the framework.
  \item Ensure Nektar++ is not fixed-dependent on additional libraries (they
  can be optionally enabled at CMake).
\end{itemize}

\section{Modifications to existing global solve}
To minimise code dependencies it is proposed to restructure the
\texttt{GlobalLinSys} class into a family of classes instantiated through a
class factory. We maintain the \texttt{GlobalLinSys} class as a base class, providing
virtual functions such as \texttt{solve()}. Additionally we provide,
for example, the following derived classes which implement linear system solvers
using various internal algorithms or external libraries (see Figure
\ref{f:globallinsys}).
\begin{itemize}
    \item GlobalLinSysDirect - a full matrix, direct solve using LAPACK.
    \item GlobalLinSysStaticCond - direct solve using substructuring
    techniques to reduce storage requirements.
    \item GlobalLinSysCG - implementation of a basic in-library conjugant
    gradient solver.
    \item GlobalLinSysPetSc - parallel iterative solver library interface.
\end{itemize}
The choice of solver to use will be selected using a \texttt{SOLVERINFO}
session file option, \texttt{GLOBAL\_SOLVE\_METHOD}, and if the specified
solver is unavailable the factory will inform the user by producing an error
message. This will encapsulate and decouple all aspects related to a
particular solve method and maintain clean code.

Additional types of solves can be implemented to provide other iterative
options. The encapsulation and code isolation allows the iterative algorithms
to utilise the NekMatrix level objects in the case of an in-house iterative
solver, or use non-native data types when utilising external libraries, as
considered appropriate.

\tikzset{my node/.code=\ifpgfmatrix\else\tikzset{matrix of nodes}\fi}

\begin{figure}
\begin{center}
\begin{tikzpicture}[node distance=2cm]%[every node/.style={my node}] 
    \node (GlobalLinSys) [class]
    {
        \textbf{GlobalLinSys}
        \nodepart{second}{Base class}
        \nodepart{third}{No dependencies}
        %\nodepart{second}
        %GlobalLinSys();\\virtual ~GlobalLinSys();\\virtual void solve();\\ 
        %\nodepart{third}{
        %    string className;
        %}
    };
    \node (GlobalLinSysDirect)[class, below of=GlobalLinSys, xshift=5cm]
    {
        \textbf{GlobalLinSysDirect}
        \nodepart{second}{Derived}
        \nodepart{third}{No dependencies}
    };
    \node (GlobalLinSysStaticCond)[class, below of=GlobalLinSysDirect]
    {
        \textbf{GlobalLinSysStaticCond}
        \nodepart{second}{Derived}
        \nodepart{third}{No dependencies}
    };
    \node (GlobalLinSysCG)[class, below of=GlobalLinSysStaticCond]
    {
        \textbf{GlobalLinSysCG}
        \nodepart{second}{Derived}
        \nodepart{third}{No dependencies}
    };
    \node (GlobalLinSysPetSc)[class, below of=GlobalLinSysCG]
    {
        \textbf{GlobalLinSysPetSc}
        \nodepart{second}{Derived}
        \nodepart{third}{PetSc Library}
    };
    \draw[myarrow] (GlobalLinSys.south) |- (GlobalLinSysDirect.west);
    \draw[myarrow] (GlobalLinSys.south) |- (GlobalLinSysStaticCond.west);
    \draw[myarrow] (GlobalLinSys.south) |- (GlobalLinSysCG.west);
    \draw[myarrow] (GlobalLinSys.south) |- (GlobalLinSysPetSc.west);
\end{tikzpicture}
\end{center}
\caption{Class structure for GlobalLinSys.}
\label{f:globallinsys}
\end{figure}


\subsection{Instantiation of GlobalLinSys objects}
At present the GlobalLinSys class is instantiated in ExpList::GenGlobalLinSys.
We propose to use the factory pattern to instantiate the required derived
GlobalLinSys class. Possible issues here are:
\begin{itemize}
  \item Currently the global matrix system are assembled before they are
  passed to the GlobalLinSys object. To resolve this one could:
  \begin{itemize}
    \item Pass the list of expansions to the GlobalLinSys object and
    encapsulate the construction within there (recommended). Unfortunately,
    this couples the GlobalLinSys class to the expansions and the underlying
    structure, although maybe this isn't of too much concern.
    \item Restructure the GlobalMatrix class in a similar way to the
    GlobalLinSys class and generate the appropriate GlobalMatrix to pass to
    the GlobalLinSys for solving. This would allow the use of arbitrary
    matrices with the GlobalLinSys solvers.
  \end{itemize}
  \textbf{Note:} GenGlobalMatrixFull and GenLinSysFull appear to duplicate
  significant portions of code. This may provide an opportunity to tidy things 
  up.
  \item In the current form, the static condensation GlobalLinSys class must
  be changed to instead receive the 4 matrices in the form of a single block
  matrix to remain constructor-compatible with the base class and Factory.
\end{itemize}

\section{Iterative Solvers}
The framework will provide a number of options for iterative solvers, allowing
the flexibility to utilise optimised external libraries and/or parallel
implementations or use an in-built conjugant gradient solver. In all cases,
each module in the framework will provide an implementation to solve a 
matrix system:
\begin{align}
\bm{Ax}=\bm{b}
\end{align}
for known $\bm{b}$ and unknown $\bm{x}$.

\subsection{Conjugant Gradient Method}
The conjugate gradient algorithm proceeds as follows:
\begin{algorithm}
%\begin{algorithmic}
$r_0 = Ax_0$ \\
$z_0 = P^{-1}r_0$ \\
$w_0 = z_0$ \\
$j=0$ \\
while not converged \\
$\alpha_j = (r_j,z_j)/(Aw_j,w_j)$\\
$x_{j+1} = x_j + \alpha_j w_j$\\
$r_{j+1} = r_j - \alpha_j w_j$\\
$z_{j+1} = P^{-1} r_{j+1}$\\
$\beta_j = (r_{j+1},z_{j+1})/(r_j,z_j$\\
$w_{j+1} = z_{j+1} + \beta_j w_j$\\
continue
%\end{algorithmic}
\end{algorithm}


\section{Dependency minimisation}
We aim to minimise the number of additional dependencies introduced.
\begin{itemize}
  \item Interfaces for external libraries should be optionally compiled and
  included through the use of CMake switches: e.g.
  \texttt{NEKTAR\_SUPPORT\_PETSC}. The corresponding \texttt{GlobalLinSys}
  derived class would only then be linked into the library.
  \item Choice of iterative solver to use should be selectable at run-time
  (use of Factory pattern and decoupled code).
\end{itemize}
\end{document}