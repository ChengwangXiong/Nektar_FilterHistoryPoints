\section{Field Convert}
\label{s:utilities:fieldconvert}
FieldConvert is a utility embedded in \nekpp with the primary 
aim of allowing the user to convert the \nekpp output binary files 
(.chk and .fld) into a format which can be read by two common 
visualisation softwares: Paraview (.vtu format) or Tecplot (.dat 
format). FieldConvert also allows the user to manipulate the 
\nekpp output binary files by using some additional modules 
which can be called with the option \inltt{-m} which stands 
for \inltt{m}odule. Note that another flag, \inltt{-r} (which stand for 
\inltt{r}ange) allows the user to specify a sub-range of the domain 
on which the conversion or manipulation of the \nekpp output binary
files will be performed. 

Almost all of the FieldConvert functionalities can be run in parallel if \nekpp 
is compiled using MPI (see the installation documentation for additional 
info on how to implement \nekpp using MPI). \footnote{the modules which 
does not have parallel support will be specified the related subsection}
%
%
%
\subsection{Convert .fld / .chk files into Paraview or Tecplot format}
\label{s:utilities:fieldconvert:sub:convert}
To convert the \nekpp output binary files (.chk and .fld) into a format 
which can be read by two common visualisation softwares: Paraview 
(.vtu format) or Tecplot (.dat format) the user can run the following 
commands:
%
\begin{itemize}
\item Paraview (.vtu format)
%
\begin{lstlisting}[style=BashInputStyle]
FieldConvert test.xml test.fld test.vtu
\end{lstlisting}
%
\item Tecplot (.dat format)
%
\begin{lstlisting}[style=BashInputStyle]
FieldConvert test.xml test.fld test.dat
\end{lstlisting}
%
\end{itemize}
%
where \inltt{FieldConvert} is the executable associated to the utility 
FieldConvert, \inltt{test.xml} is the session file and \inltt{test.dat}, 
\inltt{test.vtu} are the desired format outputs, either Tecplot or 
Paraview format respectively. 
%
\begin{tipbox}
Note that the session file is also supported 
in its compressed format \inltt{test.xml.gz}.
\end{tipbox}
%
%
%
\subsection{Field Convert range option \textit{-r}}
The Fieldconvert range option \inltt{-r} allows the user to specify 
a sub-range of the mesh (computational domain) by using an 
additional flag, \inltt{-r} (which stands for \inltt{r}ange and either 
convert or manipulate the \nekpp output binary files. 
Taking as an example the conversion of the \nekpp binary files 
(.chk or .fld) shown before and wanting to convert just the 2D 
sub-range defined by $-2\leq x \leq 3$, $-1\leq y \leq 2$ the 
additional flag \inltt{-r} can be used as follows:
%
\begin{itemize}
\item Paraview (.vtu format)
%
\begin{lstlisting}[style=BashInputStyle]
FieldConvert -r -2,3,-1,2 test.xml test.fld test.vtu
\end{lstlisting}
%
\item Tecplot (.dat format)
%
\begin{lstlisting}[style=BashInputStyle]
FieldConvert -r 2,3,-1,2 test.xml test.fld test.dat
\end{lstlisting}
%
\end{itemize}
where \inltt{-r} defines the range option of the field convert 
utility, the two first numbers define the range in $x$ direction 
and the the third and fourth number specify the $y$ range.
A sub-range of a 3D domain can also be specified. 
For doing so, a third set of numbers has to be provided 
to define the $z$ range.
%
%
%
\subsection{Field Convert modules \textit{-m}}
FieldConvert allows the user to manipulate the \nekpp output 
binary files (.chk and .fld) by using the flag \inltt{-m} (which 
stands for \inltt{m}odule).. 
Specifically, FieldConvert has these additional functionalities
%
\begin{enumerate}
\item \inltt{AddFld}: Sum two .fld files;
\item \inltt{C0Projection}: Computes the C0 projection of a given output file;
\item \inltt{QCriterion}: Computes the Q-Criterion for a given output file;
\item \inltt{concatenate}: Concatenate a \nekpp binary output (.chk or .fld) field file into single file;
\item \inltt{extract}: Extract a boundary field;
\item \inltt{interpfield}: Interpolates one field to another, requires fromxml, fromfld to be defined;
\item \inltt{interppointdatatofld}: Interpolates given discrete data using a finite difference approximation to a fld file given an xml file;
\item \inltt{interppoints}: Interpolates a set of points to another, requires fromfld and fromxml to be defined, a line or plane of points can be defined;
\item \inltt{scaleinputfld}: rescale input field by a constant factor.
\item \inltt{vorticity}: Computes the vorticity field.
\end{enumerate}
The module list above can be seen by running the command
%
\begin{lstlisting}[style=BashInputStyle]
FieldConvert -l 
\end{lstlisting}
%
In the following we will detail the usage of each module.
%
%
%

\subsubsection{Sum two .fld files: \textit{AddFld} module}
To sum two .fld files one can use the \inltt{AddFld} module of FieldConvert
%
\begin{lstlisting}[style=BashInputStyle]
FieldConvert -m addfld:fromfld=file1.fld:scale=-1 file1.xml file2.fld file3.fld
\end{lstlisting}
%
In this case we use it in conjunction with the command \inltt{scale} 
which multiply the values of a given .fld file by a constant \inltt{value}. 
\inltt{file1.fld} is the file multiplied by \inltt{value}, \inltt{file1.xml} 
is the associated session file, \inltt{file2.fld} is the .fld file which 
is summed to \inltt{file1.fld} and finally \inltt{file3.fld} is the output 
which contain the sum of the two .fld files.
\inltt{file3.fld} can be processed in a similar way as described 
in section \ref{s:utilities:fieldconvert:sub:convert} to visualise 
it either in Tecplot or in Paraview the result.
%
%
%
\subsubsection{Smooth the data: \textit{C0Projection} module}
To smooth the data of a given .fld file one can 
use the \inltt{C0Projection} module of FieldConvert
%
\begin{lstlisting}[style=BashInputStyle]
FieldConvert -m C0Projection test.xml test.fld test-C0Proj.fld
\end{lstlisting}
%
where the file \inltt{test-C0Proj.fld} can be processed in a similar 
way as described in section \ref{s:utilities:fieldconvert:sub:convert}
to visualise either in Tecplot or in Paraview the result.

\subsubsection{Calculate Q-Criterion: \textit{QCriterion} module}
To perform the Q-criterion calculation and obtain an output 
data containing the Q-criterion solution, the user can run
%
\begin{lstlisting}[style=BashInputStyle]
FieldConvert -m QCriterion test.xml test.fld test-QCrit.fld
\end{lstlisting}
%
where the file \inltt{test-QCrit.fld} can be processed in a similar 
way as described in section \ref{s:utilities:fieldconvert:sub:convert}
to visualise either in Tecplot or in Paraview the result.
%
%
%
\subsubsection{Concatenate two files: \textit{concatenate} module}
To concatenate \inltt{file1.fld} and \inltt{file2.fld} into \inltt{file-conc.fld}
one can run the following command
%
\begin{lstlisting}[style=BashInputStyle]
FieldConvert -m concatenate file.xml  file1.fld  file2.fld  file-conc.fld
\end{lstlisting}
%
where the file \inltt{file-conc.fld} can be processed in a similar 
way as described in section \ref{s:utilities:fieldconvert:sub:convert}
to visualise either in Tecplot or in Paraview the result.
%
%
%
\subsubsection{Extract a boundary region: \textit{extract} module}
The boundary region of a domain can be extracted from the output 
data using the following command line
%
\begin{lstlisting}[style=BashInputStyle]
FieldConvert -m extract:bnd=2:fldtoboundary=1 test.xml \
	test.fld test-boundary.fld
\end{lstlisting}
%
The option \inltt{bnd} specifies which boundary region to extract. 
Note this is different to MeshConvert where the parameter \inltt{surf} 
is specified and corresponds to composites rather boundaries. If \inltt{bnd} 
is not provided, all boundaries are extracted to different fields. The \inltt{fldtoboundary} 
is an optional command argument which copies the expansion of test.fld into 
the boundary region before outputting the .fld file. This option is on by default. 
If it turned off using \inltt{fldtoboundary=0} the extraction will only evaluate the 
boundary condition from the xml file. The output will be placed in test-boundary-b2.fld. 
If more than one boundary region is specified the extension -b0.fld, -b1.fld etc will be 
outputted. To process this file you will need an xml file of the same region. 
This can be generated using the command:
%
\begin{lstlisting}[style=BashInputStyle]
MeshConvert -m extract:surf=5  test.xml test-b0.xml
\end{lstlisting}
%
The surface to be extracted in this command is the composite 
number and so needs to correspond to the boundary region 
of interest. Finally to process the surface file one can use
%
\begin{lstlisting}[style=BashInputStyle]
FieldConvert  test-b0.xml test-b0.fld test-b0.dat
\end{lstlisting}
%
This will obviously generate a Tecplot output if a .dat file 
is specified as last argument. A .vtu extension will produce 
a Paraview output.
%
%
%
\subsubsection{Interpolate one field to another: \textit{interpfield} module}
To interpolate one field to another, one can use the following command:
%
\begin{lstlisting}[style=BashInputStyle]
FieldConvert -m interpfield:fromxml=file1.xml:fromfld=file1.fld \ 
	file2.xml file2.fld
\end{lstlisting}
%
This command will interpolate the field defined by \inltt{file1.xml} 
and \inltt{file1.fld} to the new mesh defined in \inltt{file2.xml} and 
output it to \inltt{file2.fld}.
The \inltt{fromxml} and \inltt{fromfld} must be specified in this module. 
In addition there are two optional arguments \inltt{clamptolowervalue} 
and \inltt{clamptouppervalue} which clamp the interpolation between 
these two values. Their default values are -10,000,000 and 10,000,000.
%
\begin{tipbox}
This module can run in parallel where the speed is increased not
only due to using more cores but also, since the mesh is split into 
smaller sub-domains, the search method currently adopted performs 
faster.
\end{tipbox}
%
%
%
\subsubsection{Interpolate scattered point data to a field: \textit{interppointdatatofld} module}
To interpolate discrete point data to a field, use the interppointdatatofld module:
%
\begin{lstlisting}[style=BashInputStyle]
FieldConvert -m interppointdatatofld file1.xml file1.pts file1.fld
\end{lstlisting}
%
This command will interpolate the data from \inltt{file1.pts} to the mesh
and expansions defined in \inltt{file1.xml} and output the field to \inltt{file1.fld}.
The file \inltt{file.pts} is of the form:
%
\begin{lstlisting}[style=XMLStyle]
<?xml version="1.0" encoding="utf-8" ?>
<NEKTAR>
  <POINTS DIM="1" FIELDS="a,b,c">
    1.0000 -1.0000 1.0000 -0.7778
    2.0000 -0.9798 0.9798 -0.7980
    3.0000 -0.9596 0.9596 -0.8182
    4.0000 -0.9394 0.9394 -0.8384
  </POINTS>
</NEKTAR>
\end{lstlisting}
%
where \inltt{DIM="1" FIELDS="a,b,c} specifies that the field is one-dimensional
and contains three variables, $a$, $b$, and $c$.
Each line defines a point, while the  first column contains its $x$-coordinate,
the second one contains the $a$-values, the third the $b$-values and so on.
In case of $n$-dimensional data, the $n$ coordinates are specified in the first $n$
columns accordingly.
Note that currently, the \textit{interppointdatatofld} module can only perform
interpolation in one dimension.
%
In order to interpolate 1D data to a $n$D field, specify the matching coordinate in
the output field using the \inltt{interpcoord} argument:
%
\begin{lstlisting}[style=BashInputStyle]
FieldConvert -m interppointdatatofld:interppointdatatofld=1 3D-file1.xml \
		1D-file1.pts 3D-file1.fld
\end{lstlisting}
%
This will interpolate the 1D scattered point data from \inltt{1D-file1.pts} to the
$y$-coordinate of the 3D mesh defined in \inltt{3D-file1.xml}. The resulting field
will have constant values along the $x$ and $z$ coordinates.
In the \inlsh{.pts}-file, all data points must be sorted by their location in ascending order.
The module implements a quadratic interpolation scheme and automatically falls back to a
linear scheme if only two data points are given.
%
%
%
\subsubsection{Interpolate a field to a series of points: \textit{interppoints} module}
You can interpolate one field to a series of given points using the following command:
\begin{lstlisting}[style=BashInputStyle]
FieldConvert -m interppoints:fromxml=from.xml:fromfld=file1.fld \
        file2.pts file2.dat
\end{lstlisting}
This command will interpolate the field defined by \inltt{file1.xml} and
\inltt{file1.fld} to the points defined in \inltt{file2.xml} and output it to
\inltt{file2.dat}.
The \inltt{fromxml} and \inltt{fromfld} must be specified in this module.
The format of the file \inltt{file2.pts} is of the same form as for the
\textit{interppointdatatofld} module:
\begin{lstlisting}[style=XMLStyle]
<?xml version="1.0" encoding="utf-8" ?>
<NEKTAR>
  <POINTS DIM="2" FIELDS="">
    0.0 0.0
    0.5 0.0
    1.0 0.0
  </POINTS>
</NEKTAR>
\end{lstlisting}
There are three optional arguments \inltt{clamptolowervalue},
\inltt{clamptouppervalue} and \inltt{defaultvalue} the first two clamp the
interpolation between these two values and the third defines the default 
value to be used if the point is outside the domain. Their default values 
are -10,000,000, 10,000,000 and 0.

In addition, instead of specifying the file \inltt{file2.pts}, a module list of the form
\begin{lstlisting}[style=BashInputStyle] 
FieldConvert -m interppoints:fromxml=file1.xml:fromfld= \ 
	file1.fld:line=npts,x0,y0,x1,y1
\end{lstlisting}
can be specified where \inltt{npts} is the number of equispaced points between
$(x0,y0)$ to $(x1,y1)$ which can also be used in 3D by specifying $(x0,y0,z0)$
to $(x1,y1,z1)$.

An extraction of a plane can also be specified by
\begin{lstlisting}[style=BashInputStyle] 
FieldConvert -m interppoints:fromxml=file1.xml:fromfld=file1.fld:plane=npts1,\
	npts2,x0,y0,z0,x1,y1,z1,x2,y2,z2,x3,y3,z3
\end{lstlisting}
where \inltt{npts1,npts2} is the number of equispaced points in each 
direction and $(x0,y0,z0)$, $(x1,y1,z1)$, $(x2,y2,z2)$ and $(x3,y3,z3)$ 
define the plane of points specified in a clockwise or anticlockwise direction.
%
\begin{notebox}
This module does not run in parallel.
\end{notebox}
%
%
%
\subsubsection{Scale a given .fld: \textit{scaleinputfld} module}

\begin{lstlisting}[style=BashInputStyle] 
FieldConvert -m sacleinputfld:scale=value  test.fld  test-scal.fld
\end{lstlisting}
The argument \inltt{scale=value} rescales of a factor \inltt{value} 
\inltt{test.fld} by the factor value.
The output file \inltt{file-conc.fld} can be processed in a similar 
way as described in section \ref{s:utilities:fieldconvert:sub:convert}
to visualise it either in Tecplot or in Paraview the result.

%
%
%
\subsubsection{Calculate vorticity: \textit{vorticity} module}
To perform the vorticity calculation and obtain an output 
data containing the vorticity solution, the user can run
\begin{lstlisting}[style=BashInputStyle]
FieldConvert -m vorticity test.xml test.fld test-vort.fld
\end{lstlisting}
where the file \inltt{test-vort.fld} can be processed in a similar 
way as described in section \ref{s:utilities:fieldconvert:sub:convert}.
%
%
%
\subsection{Field Convert in parallel}
To run FieldConvert in parallel the user needs to compile 
\nekpp with MPI support and can employ the following 
command
\begin{lstlisting}[style=BashInputStyle]
mpirun -np <nprocs> FieldConvert test.xml test.fld test.dat
\end{lstlisting}
replacing \inltt{<nprocs>} with the number of processors. 
This will produce multiple \inltt{.dat} files of the form \inltt{test\_P0.dat}, 
\inltt{test\_P1.dat}, \inltt{test\_P2.dat}. Similarly the VTK files can be
processed in this manner as can the vorticity option. In the case of 
the vorticity option a directory called \inltt{test\_vort.fld} (or the specified 
output name) will be produced with the standard parallel field files placed 
within the directory.
%
%
%
\subsection{Processing large files}
When processing large files it is not always convenient to run in parallel 
but process each parallel partition in serial, for example when interpolating 
one field to another. To do this we can use the \inltt{--nprocs} and 
\inltt{--procid} options. For example the following option will interpolate 
partition 2 of a decomposition into 10 partitions of \inltt{fiile2.xml} from 
\inltt{file1.fld}
\begin{lstlisting}[style=BashInputStyle] 
FieldConvert --nprocs 10 --procid 2 \
        -m interpfield:fromxml=file1.xml:fromfld=file1.fld \
        file2.xml file2.fld
\end{lstlisting}
This call will only provide part of the overall partition and so to create the 
full interpolated field you need to call a loop of such commands. For example 
in a bash shell you can run
\begin{lstlisting}[style=BashInputStyle] 
for n in `seq 0 1 9`
do 
    FieldConvert --nprocs 10 --procid $n \ 
            -m interpfield:fromxml=file1.xml:fromfld=file1.fld \
            file2.xml file2.fld
done
\end{lstlisting}

This will create a directory called \inltt{file2.fld} and put the different 
parallel partitions into files with names 
\inltt{P0000000.fld, P0000001.fld, ?., P0000009.fld}. This is nearly a complete
parallel field file but the Info.xml file which contains the information about 
which elements are in each partitioned file \inltt{P000000X.fld}. So to generate
this Info.xml file you need to call
\begin{lstlisting}[style=BashInputStyle] 
FieldConvert --nprocs 10 file2.xml file2.fld/Info.xml:info
\end{lstlisting}
Note the final \inltt{:info} extension on the last argument is necessary to 
tell FieldConvert that you wish to generate an info file but the extension 
to this file is .xml. This syntax allows the routine not to get confused with 
the input/output xml files.
