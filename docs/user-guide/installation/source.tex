\section{Installing from Source}
\label{s:installation:source}

This section explains how to build Nektar++ from the source-code package.

Nektar++ uses a number of third-party libraries. Some of these are required,
others are optional. It is generally more straightforward to use versions of
these libraries supplied pre-packaged for your operating system, but if you run
into difficulties with compilation errors or failing regression tests, the
Nektar++ build system can automatically build tried-and-tested versions of these
libraries for you. This requires enabling the relevant options in the CMake
configuration.


\subsection{Obtaining the source code}
There are two ways to obtain the source code for \nekpp:
\begin{itemize}
	\item Download the latest source-code archive from the
	\href{http://www.nektar.info/downloads}{Nektar++ downloads page}.
	\item Clone the git repository
	\begin{itemize}
	\item Using anonymous access. This does not require
	credentials but any changes to the code cannot be pushed to the
	public repository. Use this initially if you would like to try using 
	Nektar++.
    \begin{lstlisting}[style=BashInputStyle]
        git clone http://gitlab.nektar.info/clone/nektar.git nektar++
    \end{lstlisting}
	\item Using authenticated access. This will allow you to directly contribute
	back into the code.
    \begin{lstlisting}[style=BashInputStyle]
        git clone git@gitlab.nektar.info:nektar.git nektar++
    \end{lstlisting}
    \begin{tipbox}
    You can easily switch to using the authenticated access from anonymous
    access at a later date.
    \end{tipbox}
	\end{itemize}
\end{itemize}

\subsection{Linux}
\subsubsection{Prerequisites}
\nekpp uses a number of external programs and libraries for some or all of its
functionality. Some of these are \emph{required} and must be installed prior to
compiling Nektar++, most of which are available as pre-built \emph{system}
packages on most Linux distributions or can be installed manually by a
\emph{user}. Others are optional and required only for specific features, or can
be downloaded and compiled for use with Nektar++ \emph{automatically} (but not
installed system-wide).

\begin{center}
\begin{tabular}{lccccl}
\toprule
Package & Req. & \multicolumn{3}{c}{Installation} & Note \\ \cline{3-5}
        &      & Sys. & User & Auto.              & \\
\midrule
C++ compiler    & \checkmark & \checkmark & & & gcc, icc, etc \\
CMake  $>2.8.7$ & \checkmark & \checkmark & \checkmark &            & Ncurses
GUI optional
\\
BLAS            & \checkmark & \checkmark & \checkmark &            & Or MKL,
ACML, OpenBLAS
\\
LAPACK          & \checkmark & \checkmark & \checkmark &            & \\
Boost $>1.49$   & \checkmark & \checkmark & \checkmark & \checkmark & Compile
with iostreams
\\
ModMETIS        & \checkmark &            &            & \checkmark & \\
FFTW $>3.0$     &            & \checkmark & \checkmark & \checkmark & For
high-performance FFTs\\
ARPACK $>2.0$   &            & \checkmark & \checkmark &            & For
arnoldi algorithms\\
OpenMPI         &            & \checkmark &            &            & For
parallel execution\\
GSMPI           &            &            &            & \checkmark & For
parallel execution\\
PetSC           &            &            & \checkmark & \checkmark &
Alternative linear solvers\\
Scotch          &            & \checkmark & \checkmark & \checkmark &
Alternative mesh partitioning\\
VTK $>5.8$      &         & \checkmark & \checkmark &            & Visualisation
utilities\\
\bottomrule
\end{tabular}
\end{center}

\begin{warningbox}
Boost version 1.51 has a bug which prevents \nekpp working correctly.
Please use a newer version.
\end{warningbox}


\subsubsection{Quick Start}
Open a terminal.

If you have downloaded the tarball, first unpack it:
\begin{lstlisting}[style=BashInputStyle]
    tar -zxvf nektar++-4.0.0.tar.gz
\end{lstlisting}
Change into the \inlsh{nektar++} source code directory
\begin{lstlisting}[style=BashInputStyle]
    mkdir -p build && cd build 
    ccmake ../
    make install
\end{lstlisting}

\subsubsection{Detailed instructions}
From a terminal:
\begin{enumerate}
    \item If you have downloaded the tarball, first unpack it
    \begin{lstlisting}[style=BashInputStyle]
    tar -zxvf nektar++-4.0.0.tar.gz
    \end{lstlisting}
    
    \item Change into the source-code directory, create a \inltt{build}
    subdirectory and enter it 
    \begin{lstlisting}[style=BashInputStyle]
    mkdir -p build && cd build
    \end{lstlisting}
    
    \item Run the CMake GUI and configure the build
    \begin{lstlisting}[style=BashInputStyle]
    ccmake ../
    \end{lstlisting}
    \begin{itemize}
        \item Select the components of Nektar++ (prefixed with
        \inltt{NEKTAR\_BUILD\_}) you would like to build. Disabling solvers
        which you do not require will speed up the build process.
        \item Select the optional libraries you would like to use (prefixed with
        \inltt{NEKTAR\_USE\_}) for additional functionality.
        \item Select the libraries not already available on your system which
        you wish to be compiled automatically (prefixed with 
        \inltt{THIRDPARTY\_BUILD\_})
    \end{itemize}
    A full list of configuration options can be found in
    Section~\ref{s:installation:source:cmake}.

    \begin{notebox}
    Selecting \inltt{THIRDPARTY\_BUILD\_} options will request CMake to
    automatically download thirdparty libraries and compile them within the
    \nekpp directory. If you have administrative access to your machine, it is
    recommended to install the libraries system-wide through your
    package-management system.
    \end{notebox}
    
    
    \item Press \inltt{c} to configure the build. If errors arise relating to
    missing libraries, review the \inltt{THIRDPARTY\_BUILD\_} selections in
    the configuration step above or install the missing libraries manually or
    from system packages.
    
    \item When configuration completes without errors, press \inltt{c} again
    until the option \inltt{g} to generate build files appears. Press \inltt{g}
    to generate the build files and exit CMake.
    
    \item Compile the code
    \begin{lstlisting}[style=BashInputStyle]
        make install
    \end{lstlisting}
    During the build, missing third-party libraries will be automatically
    downloaded, configured and built in the \nekpp \inlsh{build} directory.
    
    % Hacky way to get an lstlisting to an argument of a macro
    \newsavebox\installationLinuxTip
    \begin{lrbox}{\installationLinuxTip}\begin{minipage}{0.8\linewidth}
    \begin{lstlisting}[style=BashInputStyle]
    make -j4 install
    \end{lstlisting}
    \end{minipage}
    \end{lrbox}
    
    \begin{tipbox}
    If you have multiple processors/cores on your system, compilation can be
    significantly increased by adding the \inlsh{-jX} option to make, where X is
    the number of simultaneous jobs to spawn. For example, use
    
    \noindent\usebox\installationLinuxTip
    
    on a quad-core system.
    \end{tipbox}
    
    \item Test the build by running unit and regression tests.
    \begin{lstlisting}[style=BashInputStyle]
    ctest
    \end{lstlisting}
\end{enumerate}

\subsection{OSX}

\subsubsection{Prerequisites}
\nekpp uses a number of external programs and libraries for some or all of its
functionality. Some of these are \emph{required} and must be installed prior to
compiling Nektar++, most of which are available on \emph{MacPorts}
(www.macports.org) or can be installed manually by a \emph{user}. Others are
optional and required only for specific features, or can be downloaded and 
compiled for use with Nektar++ \emph{automatically} (but not installed
system-wide).

\begin{notebox}
To compile \nekpp on OSX, Apple's Xcode Developer Tools must be installed. They
can be installed either from the App Store (only on Mac OS 10.7 and above) or
downloaded directly from
\href{http://connect.apple.com/}{http://connect.apple.com/} 
(you are required to have an Apple Developer Connection account).
Xcode includes Apple implementations of BLAS and LAPACK (called the Accelerate
Framework).
\end{notebox}

\begin{center}
\begin{tabular}{lccccl}
\toprule
Package & Req. & \multicolumn{3}{c}{Installation} & Note \\ \cline{3-5}
        &      & MacPorts & User & Auto.              & \\
\midrule
Xcode           & \checkmark &            & & & gcc, icc, etc \\
CMake  $>2.8.7$ & \checkmark & \checkmark & \checkmark &            & Ncurses
GUI optional \\
BLAS            & \checkmark &            &            &            & Part of
Xcode \\
LAPACK          & \checkmark &            &            &            & Part of
Xcode \\
Boost $>1.49$   & \checkmark & \checkmark & \checkmark & \checkmark & Compile
with iostreams
\\
ModMETIS        & \checkmark &            &            & \checkmark & \\
FFTW $>3.0$     &            & \checkmark & \checkmark & \checkmark & For
high-performance FFTs\\
ARPACK $>2.0$   &            & \checkmark & \checkmark &            & For
arnoldi algorithms\\
OpenMPI         &            & \checkmark &            &            & For
parallel execution\\
GSMPI           &            &            &            & \checkmark & For
parallel execution\\
PetSC           &            & \checkmark & \checkmark & \checkmark &
Alternative linear solvers\\
Scotch          &            &            & \checkmark & \checkmark &
Alternative mesh partitioning\\
VTK $>5.8$      &            & \checkmark & \checkmark &            &
Visualisation utilities\\
\bottomrule
\end{tabular}
\end{center}

\newsavebox\installationOSXMacPortsTip
\begin{lrbox}{\installationOSXMacPortsTip}\begin{minipage}{0.8\linewidth}
\begin{lstlisting}[style=BashInputStyle]
sudo port install cmake
\end{lstlisting}
\end{minipage}
\end{lrbox}

\begin{tipbox}
CMake, and some other software, is available from MacPorts
(\url{http://macports.org}) and can be installed using, for example,

\noindent\usebox\installationOSXMacPortsTip
\end{tipbox}


\subsubsection{Quick Start}
Open a terminal (Applications->Utilities->Terminal).

If you have downloaded the tarball, first unpack it:
\begin{lstlisting}[style=BashInputStyle]
    tar -zxvf nektar++-4.0.0.tar.gz
\end{lstlisting}
Change into the \inlsh{nektar++} source code directory
\begin{lstlisting}[style=BashInputStyle]
    mkdir -p build && cd build 
    ccmake ../
    make install
\end{lstlisting}

\subsubsection{Detailed instructions}
From a terminal (Applications->Utilities->Terminal):
\begin{enumerate}
    \item If you have downloaded the tarball, first unpack it
    \begin{lstlisting}[style=BashInputStyle]
    tar -zxvf nektar++-4.0.0.tar.gz
    \end{lstlisting}

    \item Change into the source-code directory, create a \inltt{build}
    subdirectory and enter it 
    \begin{lstlisting}[style=BashInputStyle]
    mkdir -p build && cd build
    \end{lstlisting}
    
    \item Run the CMake GUI and configure the build
    \begin{lstlisting}[style=BashInputStyle]
    ccmake ../
    \end{lstlisting}
    Use the arrow keys to navigate the options and \inlsh{ENTER} to select/edit
    an option.
    \begin{itemize}
        \item Select the components of Nektar++ (prefixed with
        \inltt{NEKTAR\_BUILD\_}) you would like to build. Disabling solvers
        which you do not require will speed up the build process.
        \item Select the optional libraries you would like to use (prefixed with
        \inltt{NEKTAR\_USE\_}) for additional functionality.
        \item Select the libraries not already available on your system which
        you wish to be compiled automatically (prefixed with 
        \inltt{THIRDPARTY\_BUILD\_})
        \item 
    \end{itemize}
    A full list of configuration options can be found in
    Section~\ref{s:installation:source:cmake}.

    \begin{notebox}
    Selecting \inltt{THIRDPARTY\_BUILD\_} options will request CMake to
    automatically download thirdparty libraries and compile them within the
    \nekpp directory. If you have administrative access to your machine, it is
    recommended to install the libraries system-wide through MacPorts.
    \end{notebox}
    
    \item Press \inltt{c} to configure the build. If errors arise relating to
        missing libraries (variables set to \inlsh{NOTFOUND}), review the 
        \inltt{THIRDPARTY\_BUILD\_} selections in the previous
    step or install the missing libraries manually or through MacPorts. 
    
    \item When configuration completes without errors, press \inltt{c} again
    until the option \inltt{g} to generate build files appears. Press \inltt{g}
    to generate the build files and exit CMake.
    
    \item Compile the code
    \begin{lstlisting}[style=BashInputStyle]
        make install
    \end{lstlisting}
    During the build, missing third-party libraries will be automatically
    downloaded, configured and built in the \nekpp \inlsh{build} directory.
    
    % Hacky way to get an lstlisting to an argument of a macro
    \newsavebox\installationMacTip
    \begin{lrbox}{\installationMacTip}\begin{minipage}{0.8\linewidth}
    \begin{lstlisting}[style=BashInputStyle]
    make -j4 install
    \end{lstlisting}
    \end{minipage}
    \end{lrbox}
    
    \begin{tipbox}
    If you have multiple processors/cores on your system, compilation can be
    significantly increased by adding the \inlsh{-jX} option to make, where X is
    the number of simultaneous jobs to spawn. For example,
    \noindent\usebox\installationMacTip
    \end{tipbox}
    
    \item Test the build by running unit and regression tests.
    \begin{lstlisting}[style=BashInputStyle]
    ctest
    \end{lstlisting}
\end{enumerate}



\subsection{Windows}
The following software must be pre-installed on your system:
\begin{itemize}
\item Microsoft Visual Studio 2008
\begin{itemize} 
    \item Visual Studio 2008 Service Pack 1 is required
\end{itemize}
\item Python Compiler 2.7.2+ (download from
\url{http://www.http//python.org/download/}) 
\item CMake 2.8.7+ (download from \url{http://www.cmake.org/HTML/index.html})
\begin{itemize}
    \item When prompted, select the option to add CMake to the system PATH.
\end{itemize}
\item WinRAR (download from \url{http://www.rarlab.com/download.htm})
\end{itemize}

\newcommand{\nekver}{\input{../../VERSION}}
\subsubsection{Unpack the tarballs}
\begin{itemize}
\item Unpack the \inlsh{nektar++-\nekver.tar.gz} file using
WinRAR
\item Create a \inlsh{ThirdParty} directory within the
\inlsh{nektar++-\nekver} subdirectory.

\begin{notebox}
 Some windows version do not
 recognise the path of a folder which has '++' in the name. If you think that 
 your windows version can not handle path containing special characters, you 
 should rename \inlsh{nektar++-\nekver} to
 \inlsh{nektar-\nekver}.
 \end{notebox}
 
\end{itemize}

\subsubsection{Building ThirdParty libraries}
Download ThirdParty packages from \inlsh{www.nektar.info/thirdparty} as requried
below.
\begin{itemize}
\item \textbf{TinyXML} Extract the \inlsh{tinyxml\_2\_4\_3.zip} archive into
the ThirdParty directory using WinRAR. This will create a \inlsh{tinyxml}
sub-directory.
\item \textbf{Loki} Extract the \inlsh{loki-0.1.3.tar.bz2} archive into the
ThirdParty directory using WinRAR. This will create a \inlsh{loki-0.1.3}
sub-directory.
\item \textbf{Mod-METIS} Extract the \inlsh{modmetis-4.0.tar.bz2} archive
into the ThirdParty directory using WinRAR. This will create a
\inlsh{modmetis-4.0} sub-directory.
\begin{itemize}
  \item Open a command-line terminal and change to
   \inlsh{ThirdParty\textbackslash modmetis-4.0\textbackslash build}.
  \item Run
    \begin{lstlisting}[style=BashInputStyle]
    cmake ..\
    vcbuild "ModifiedMetis.sln"
    vcbuild "INSTALL.vcproj"
    \end{lstlisting}
\end{itemize}
\item \textbf{NIST SparseBLAS} Extract the \inlsh{spblastk0.9b} archive
into the ThirdParty directory using WinRAR. This will create a
\inlsh{spblastk0.9b} sub-directory.
\begin{itemize}
  \item From a command-line terminal change to
  \inlsh{ThirdParty\textbackslash spblastk0.9b\textbackslash build}.
  \item Run
    \begin{lstlisting}[style=BashInputStyle]
    cmake ..\
    vcbuild "NistSparseBlasToolkit.sln"
    vcbuild "INSTALL.vcproj"
    \end{lstlisting}
\end{itemize}
\item \textbf{BLAS/LAPACK} Extract the \inlsh{blaslapack-dll.zip} archive
into the ThirdParty directory using WinRAR. This will create
\inlsh{blas\_win32.*} and \inlsh{lapack\_win32.*} files in the ThirdParty
directory.

\item \textbf{ZLib} Extract the \inlsh{zlib-1.2.3.tar.bz2} archive using
WinRAR. This will create a \inlsh{zlib-1.2.3} sub-directory.

\item \textbf{Boost} Extract the \inlsh{boost\_1\_49\_0.tar.bz2} archive using
WinRAR. This will create a {{{boost\_1\_49\_0}}} sub-directory.
\begin{itemize}
  \item Open a command-line terminal and change to the \inlsh{boost\_1\_49\_0}
  directory and run:
    \begin{lstlisting}[style=BashInputStyle]
    bootstrap.bat
    b2.exe -s ZLIB_SOURCE=..\..\..\..\zlib-1.2.3 \ 
            --prefix=..\boost toolset=msvc-9.0 install
    \end{lstlisting}
\end{itemize}
\begin{notebox}
Boost might not be able to find the \inlsh{zlib-1.2.3} directory specified
by the relative path in the command line above. If this happens, you can 
replace the relative path with the absolute path of the \inlsh{zlib-1.2.3}
directory (surround it in quotes).
\end{notebox}
\end{itemize}

\subsubsection{Building Nektar++}
Change to the \inlsh{nektar++-\nekver\textbackslash builds} directory.
\begin{lstlisting}[style=BashInputStyle]
cmake ..\
vcbuild "Nektar++.sln"
vcbuild "INSTALL.vcproj"
\end{lstlisting}
To modify the default configuration use \inlsh{cmake -i ..\textbackslash} or use
the CMake-gui application.


\subsubsection{Notes on building on windows 7 64 bit with Visual Studio 10}
To get a build working on a windows 7 64 bit machine with VS10, the following
modifications were helpful or necessary:
\begin{itemize}
\item Instead of vcbuild (which doesn't exist in VS10) use msbuild. 
\item You may need to run \\ \inlsh{C:\textbackslash Program Files
(x86)\textbackslash Microsoft Visual Studio 10.0\textbackslash
VC\textbackslash vcvarsall.bat} \\ first to set up some registry entries to run
msbuild.
 Try without first.
\item You will probably wish to put msbuild into your path
\item project files created by cmake are likely to have the extension .vcxproj
 rather than .vcproj.
\item When running b2.exe, provide the full path to zlib. Not doing so will only
 cause an error later when running cmake for Nektar++ 
\item When running b2.exe, use toolset=msvc-10.0 instead of toolset=msvc-9.0
\item When using cmake for nektar++, it is helpful to use the cmake gui
\item Using the cmake gui, set the BOOST\_ROOT variable to the appropriate path
for boost * Set Boost\_USE\_STATIC\_LIBS = ON (I did this by editing
CMakeLists.txt, but setting it from the gui should work)
\item When running msbuild "Nektar++.sln" I got error can't find zlib.lib. As a
 workaround, copy the files zlibd.lib (two of them) and rename the copies zlib.lib. 
 However, I believe these are the debug builds, so it would be better to work 
 out how to build in release mode.
\item At this stage, when running msbuild "Nektar++.sln", I got boost link
errors.
 To fix them, look for the line
\begin{lstlisting}[style=BashInputStyle]
ADD_DEFINITIONS(-DBOOST_ALL_NO_LIB -DBOOST_PROGRAM_OPTIONS_DYN_LINK \
    -DBOOST_IOSTREAMS_DYN_LINK -DBOOST_THREAD_DYN_LINK)
\end{lstlisting}
in the \inlsh{CMakeLists.txt} file and comment it out (and run cmake again) 
\item At this point, when building Nektar, linking against boost is fine, but I
got a compile error in TestData.cpp. This could be fixed by editing TestData.cpp
and replacing the line 
\begin{lstlisting}
m_doc = new TiXmlDocument(pFilename.file_string().c_str());
\end{lstlisting}
with
\begin{lstlisting}
m_doc =new TiXmlDocument(pFilename.string().c_str());
\end{lstlisting}
\end{itemize}

\begin{notebox}
The step of building INSTALL.vcproj causes executables and dlls to be copied
into a directory \inlsh{<path>\textbackslash
nektar\textbackslash builds-test\textbackslash dist\textbackslash bin}.
The executables should be run from here so that they can find the dlls they need.
\end{notebox}

If you wish to put a breakpoint in to step through Nektar++, you may need to run
from the folder the executable was built in so that the pdb files are in the
expected place. Then a workaround is to copy the dlls into that folder.

\subsection{CMake Option Reference}
\label{s:installation:source:cmake}
This section describes the main configuration options which can be set when
building Nektar++. The default options should work on almost all systems, but
additional features (such as parallelisation and specialist libraries) can be
enabled if needed.

\subsubsection{Components}
The first set of options specify the components of the Nektar++ toolkit to
compile. Some options are dependent on others being enabled, so the available
options may change.

Components of the \nekpp package can be selected using the following options:
\begin{itemize}
    \item \inlsh{NEKTAR\_BUILD\_DEMOS} (Recommended)
    
    Compiles the demonstration programs. These are primarily used by the
    regression testing suite to verify the \nekpp library, but also provide an
    example of the basic usage of the framework.

    \item \inlsh{NEKTAR\_BUILD\_DOC}

    Compiles the Doxygen documentation for the code. This will be put in
    \begin{lstlisting}[style=BashInputStyle]
    $BUILDDIR/doxygen/html
    \end{lstlisting}

    \item \inlsh{NEKTAR\_BUILD\_LIBRARY} (Required)
    
    Compiles the Nektar++ framework libraries. This is required for all other
    options.

    \item \inlsh{NEKTAR\_BUILD\_SOLVERS} (Recommended)
    
    Compiles the solvers distributed with the \nekpp framework.

    If enabling \inlsh{NEKTAR\_BUILD\_SOLVERS}, individual solvers can be
    enabled or disabled. See Chapter~\ref{s:solvers} for the list of available
    solvers. You can disable solvers which are not required to reduce
    compilation time. See the \inlsh{NEKTAR\_SOLVER\_X} option.

    \item \inlsh{NEKTAR\_BUILD\_TESTS} (Recommended)

    Compiles the testing program used to verify the \nekpp framework.

    \item \inlsh{NEKTAR\_BUILD\_TIMINGS}

    Compiles programs used for timing \nekpp operations.

    \item \inlsh{NEKTAR\_BUILD\_UNIT\_TESTS}

    Compiles tests for checking the core library functions.

    \item \inlsh{NEKTAR\_BUILD\_UTILITIES}

    Compiles utilities for pre- and post-processing simulation data.
    
    \item \inlsh{NEKTAR\_SOLVER\_X}
    
    Enabled compilation of the 'X' solver.
\end{itemize}

A number of ThirdParty libraries are required by \nekpp. There are also
optional libraries which provide additional functionality. These can be selected
using the following options:
\begin{itemize}
    \item \inlsh{NEKTAR\_USE\_BLAS\_LAPACK} (Required)
    
    Enables the use of Basic Linear Algebra Subroutines libraries for linear
    algebra operations.

    \item \inlsh{NEKTAR\_USE\_SYSTEM\_BLAS\_LAPACK} (Recommended)
    
    On Linux systems, use the default BLAS and LAPACK library on the system.
    This may not be the implementation offering the highest performance for your
    architecture, but it is the most likely to work without problem.
    
    \item \inlsh{NEKTAR\_USE\_OPENBLAS}
    
    Use OpenBLAS for the BLAS library. OpenBLAS is based on the Goto BLAS
    implementation and generally offers better performance than a non-optimised
    system BLAS. However, the library must be installed on the system.
    
    \item \inlsh{NEKTAR\_USE\_MKL}
    
    Use the Intel MKL library. This is typically available on cluster
    environments and should offer performance tuned for the specific cluster
    environment.
    
    \item \inlsh{NEKTAR\_USE\_MPI} (Recommended)
    
    Build Nektar++ with MPI parallelisation. This allows solvers to be run in
    serial or parallel.
    
    \item \inlsh{NEKTAR\_USE\_FFTW}
    
    Build Nektar++ with support for FFTW for performing Fast Fourier Transforms
    (FFTs). This is used only when using domains with homogeneous coordinate
    directions.
    
    \item \inlsh{NEKTAR\_USE\_ARPACK}
    
    Build Nektar++ with support for ARPACK. This provides routines used for
    linear stability analyses. Alternative Arnoldi algorithms are also
    implemented directly in Nektar++.
    
    \item \inlsh{NEKTAR\_USE\_VTK}
    
    Build Nektar++ with support for VTK libraries. This is only needed for
    specialist utilities and is not needed for general use.
    
    \begin{notebox}
    The VTK libraries are not needed for converting the output of simulations to
    VTK format for visualization as this is handled internally.
    \end{notebox}
\end{itemize}

The \inlsh{THIRDPARTY\_BUILD\_X} options select which third-party libraries are
automatically built during the \nekpp build process. Below are the choices of X:
\begin{itemize}
    \item \inlsh{BOOST}

    The \emph{Boost} libraries are frequently provided by the operating system,
    so automatic compilation is not enabled by default. If you do not have
    Boost on your system, you can enable this to have Boost configured
    automatically.

    \item \inlsh{GSMPI}
    
    (MPI-only) Parallel communication library.
    
    \item \inlsh{LOKI}
    
    An implementation of a singleton.
    
    \item \inlsh{METIS}
    
    A graph partitioning library used for substructuring of matrices and mesh
    partitioning when Nektar++ is run in parallel.
    
    \item \inlsh{TINYXML}
    
    Library for reading and writing XML files.
\end{itemize}

