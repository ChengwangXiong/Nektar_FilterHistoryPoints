\section{Installing Debian/Ubuntu Packages}
\label{s:installation:debian}
Binary packages are available for current Debian/Ubuntu based Linux
distributions. These can be installed through the use of standard system package
management utilities such as Apt if administrative access is
available.

\begin{enumerate}
	\item Add the appropriate line for the Debian-based distribution to the end of
	the file \inlsh{/etc/apt/sources.list}
	
	{\small
	\begin{tabular}{ll}
	\toprule
	Distribution & Repository \\
	\midrule
	Debian 7.0 (wheezy) & 
	   \texttt{deb http://www.nektar.info/debian wheezy contrib} \\
	Ubuntu 14.04 (trusty) & 
	   \texttt{deb http://www.nektar.info/ubuntu trusty contrib}\\
	\bottomrule
	\end{tabular}
	}
	\item Update the package lists
	\begin{lstlisting}[style=BashInputStyle]
	apt-get update
	\end{lstlisting}
	\item Install the required Nektar++ packages:
	\begin{lstlisting}[style=BashInputStyle]
	apt-get install nektar++
	\end{lstlisting}
	Any additional dependencies required for Nektar++ to function will be
	automatically installed.
	
	% Hacky way to get an lstlisting to an argument of a macro
    \newsavebox\installationDebTip
    \begin{lrbox}{\installationDebTip}\begin{minipage}{0.8\linewidth}
    \begin{lstlisting}[style=BashInputStyle]
    apt-cache search nektar++
    \end{lstlisting}
    \end{minipage}
    \end{lrbox}
	
	\begin{tipbox}
	Nektar++ is split into multiple packages for the different components of the
	software. A list of available Nektar++ packages can be found using:
	\noindent\usebox\installationDebTip
	\end{tipbox}
\end{enumerate}


\section{Installing Redhat/Fedora Packages}
\label{s:installation:redhat}
Add a file to the directory \inlsh{/etc/yum.repos.d/nektar.repo} with the
following contents
\begin{lstlisting}[style=BashInputStyle]
[Nektar]
name=nektar
\end{lstlisting}

Add a line specifying the \inlsh{baseurl}, depending on the version of your
distribution:

{\small
\begin{center}
\begin{tabular}{ll}
\toprule
Distribution & baseurl \\
\midrule
Fedora 20 & 
   \texttt{http://ww.nektar.info/fedora/20/\$basearch}\\
\bottomrule
\end{tabular}
\end{center}
}

\begin{notebox}
The \inltt{\$basearch} variable is replaced automatically by Yum with
the architecture of your system.
\end{notebox}



% \section{Installing OSX Packages}
% \label{s:installation:osx}
