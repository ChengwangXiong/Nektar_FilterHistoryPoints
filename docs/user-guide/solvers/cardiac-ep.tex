\chapter{Cardiac Electrophysiology Solver}

\section{Synopsis}

The CardiacEPSolver is intended to model the electrophysiology of cardiac
tissue, specifically using the monodomain or bidomain model. These models are
continuum models and represent an average of the electrical activity over many
cells. The system is a reaction-diffusion system, with the reaction term
modeling the flow of current in and out of the cells using a separate set of
ODEs.

\subsection{Bidomain Model}
The Bidomain model is given by the following PDEs,
\begin{align*}
g_{ix}\frac{\partial^2 V_i}{\partial x^2} + g_{iy}\frac{\partial^2 V_i}{\partial y^2} &=  \chi \left[ C_m \frac{\partial(V_i-V_e)}{\partial t} + G_m(V_i-V_e) \right] \\
g_{ex}\frac{\partial^2 V_e}{\partial x^2} + g_{ey}\frac{\partial^2 V_e}{\partial y^2} &= -\chi\left[ C_m \frac{\partial(V_i-V_e)}{\partial t} + G_m(V_i-V_e) \right].
\end{align*}
However, when solving numerically, one often rewrites these equations in terms
of the transmembrane potential and extracellular potential,
\begin{align*}
\chi \left[ C_m \frac{\partial V_m}{\partial t} + J_{ion} \right] &= g_{ex}\frac{\partial^2 V_e}{\partial x^2} + g_{ey}\frac{\partial^2 V_e}{\partial y^2}\\
(g_{ix} + g_{ex})\frac{\partial^2 V_e}{\partial x^2} + (g_{iy} + g_{ey})\frac{\partial^2 V_e}{\partial y^2} 
  &= -g_{ix} \frac{\partial^2 V_m}{\partial x^2} - g_{iy} \frac{\partial^2 V_m}{\partial y^2}
\end{align*}

\subsection{Monodomain Model}
In the case where the intracellular and extracellular conductivities are
proportional, that is $g_{ix} = kg_{ex}$ for some $k$,
then the above two PDEs can be reduced to a single PDE:
\begin{align*}
\chi\left[ C_m \frac{\partial V_m}{\partial t} + J_{ion} \right] &= \nabla \cdot (\sigma \nabla V_m)
\end{align*}

\subsection{Cell Models}
The action potential of a cardiac cell can be modelled at either a biophysical
level of detail, including a number of transmembrane currents, or as a
phenomenological model, to reproduce the features of the action potential, with
fewer variables. Each cell model will include a unique system of ODEs to
represent the gating variables of that model.

A number of ionic cell models are currently supported by the solver including:
\begin{itemize}
    \item Courtemanche, Ramirez, Nattel, 1998 
    \item Luo, Rudy, 1991
    \item ten Tusscher, Panfilov, 2006 (epicardial, endocardial and
    mid-myocardial variants)
\end{itemize}

Phenomological cell models are also supported:
\begin{itemize}
    \item Aliev-Panfilov
    \item Fitzhugh-Nagumo
\end{itemize}

It is important to ensure that the units of the voltage and currents from the
cell model are consistent with the units expected by the tissue level solver
(monodomain/bidomain). We will show as an example the Courtemanche, Ramirez,
Nattel, 1998 human atrial model.

The monodomain equation:
\begin{align*}
\chi \left[ C_m \frac{\partial V_m}{\partial t} + J_{ion} \right] &= \nabla \cdot (\sigma \nabla V_m)
\end{align*}

\section{Usage}

\begin{lstlisting}[style=BashInputStyle]
CardiacEPSolver session.xml
\end{lstlisting}

\section{Session file configuration}
\subsection{Solver Info}
\begin{itemize}
    \item \inltt{Eqtype} Specifies the PDE system to solve. This should take
    the value \inltt{Monodomain}.
    \item \inltt{CellModel} Specifies the cell model to use. Available cell
    models are

    \begin{center}
    \begin{tabular}{l|l|l|l}
    Value & Description & No. of Var. & Ref. \\
    \hline
    \inltt{AlievPanfilov} & Phenomological & 1 & \cite{AlPa96} \\
    \inltt{CourtemancheRamirezNattel98} & Human atrial & 20 & \cite{CoRaNa98} \\
    \inltt{FitzHughNagumo} & & & \\
    \inltt{Fox02} & & & \\
    \inltt{LuoRudy91} & Mammalian ventricular & 7 & \cite{LuRu91} \\
    \inltt{PanditGilesDemir03} & & & \\
    \inltt{TenTusscher06} & Human ventricular & 18 & \cite{TuPa06} \\
    \inltt{Winslow99} & & & \\
    \hline
    \end{tabular}
    \end{center}
    
    \item \inltt{Projection} Specifies the Galerkin projection type to use. Only
\inltt{Continuous} has been extensively tested.
    \item \inltt{TimeIntegrationMethod} Specifies the time integration
    scheme to use for advancing the PDE system. This must be an IMEX scheme. 
    Suitable choices are: \inltt{IMEXOrder1}, \inltt{IMEXOrder2},
    \inltt{IMEXOrder3}, \inltt{IMEXdirk\_3\_4\_3}. The cell model state
    variables are time advanced using Forward Euler for the ion concentrations, and 
    Rush-Larsen for the cell model gating variables.
    \item \inltt{DiffusionAdvancement} Specifies whether the diffusion is
    handled implicitly or explicitly in the time integration scheme. The current
    code only supports \inltt{Implicit} integration of the diffusion term. The
    cell model is always integrated explicitly.
\end{itemize}


\subsection{Parameters}
The following parameters can be specified in the \inltt{PARAMETERS} section of
the session file. Example values are taken from \cite{NiKeBeBeBe11}.
\begin{itemize}
    \item \inltt{Chi} sets the surface-to-volume ratio (Units:
 $\mathrm{mm}^{-1}$).\\ Example: $\chi= 140 \mathrm{mm}^{-1}$
    \item \inltt{Cm} sets the specific membrane capacitance (Units:
 $\mathrm{\mu F\,mm}^{-2}$).\\ Example: $C_m= 0.01 \mathrm{\mu F\,mm}^{-2}$
    \item \inltt{Substeps} sets the number of substeps taken in time
    integrating the cell model for each PDE timestep.\\ Example: 4
    \item \inltt{d\_min}, \inltt{d\_max}, \inltt{o\_min}, \inltt{o\_max}
    specifies a bijective map to assign conductivity values $\sigma$ to
    intensity values $\mu$ when using the \inltt{IsotropicConductivity}
    function. The intensity map is first thresholded to the range
    $[d_{\min},d_{\max}]$ and then the conductivity is calculated as
\begin{align*}
\sigma = \frac{o_{\max} - o_{\min}}{d_{\max}-d_{\min}} (1 - \mu) + o_{\min}
\end{align*}
\end{itemize}

\subsection{Functions}
The following functions can be specified inside the \inltt{CONDITIONS} section
of the session file. If both are specified, the effect is multiplicative.
Example values are taken from \cite{NiKeBeBeBe11}.
\begin{itemize}
	\item \inltt{IsotropicConductivity} specifies the conductivity
	$\sigma$ of the tissue. \\ 
	Example: $\sigma =0.13341 \ \mathrm{mS\,mm}^{-1}$, based on $\sigma
	= \frac{\sigma_i \sigma_e} {\sigma_i + \sigma_e}, \ \  \sigma_i=0.17, \sigma_e=0.62  \mathrm{mS\,mm}^{-1} $
  
  The variable name to use is \inltt{intensity} since the conductivity may be
  derived from late-Gadolinium enhanced MRA imaging. Example specifications are 
\begin{lstlisting}[style=XmlStyle]
<E VAR="intensity" VALUE="0.13341" />
<F VAR="intensity" FILE="scarmap.con" />
\end{lstlisting}
  where \inltt{scarmap.con} is a Nektar++ field file containing a variable
  \inltt{intensity} describing the conductivity across the domain.

    \item \inltt{AnisotropicConductivity} specifies the conductivity
$\mathbf{\sigma}$ of the tissue.
\end{itemize}


\subsection{Filters}
\begin{itemize}
    \item \inltt{CheckpointCellModel} checkpoints the cell model. Can be
    used along with the \inltt{Checkpoint Filter} to record complete simulation
    state and regular intervals.
\begin{lstlisting}[style=XmlStyle]
<FILTER TYPE="CheckpointCellModel">
  <PARAM NAME="OutputFile"> session </PARAM>
  <PARAM NAME="OutputFrequency"> 1 </PARAM>
</FILTER>
\end{lstlisting}
    \begin{itemize}
        \item \inltt{OutputFile} (optional) specifies the base filename to use.
    If not specified, the session name is used. Checkpoint files are suffixed with the process ID and the extension `.chk`.
        \item \inltt{OutputFrequency} specifies the number of timesteps between
    checkpoints.
    \end{itemize}
    
    \item \inltt{Electrogram} Computes virtual unipolar electrograms at a
     prescribed set of points.
\begin{lstlisting}[style=XmlStyle]
<FILTER TYPE="Electrogram">
  <PARAM NAME="OutputFile"> session </PARAM>
  <PARAM NAME="OutputFrequency"> 1 </PARAM>
  <PARAM NAME="Points">
      0.0  0.5  0.7
      1.0  0.5  0.7
      2.0  0.5  0.7
  </PARAM>
</FILTER>
\end{lstlisting}
    \begin{itemize}
    \item \inltt{OutputFile} (optional) specifies the base filename to use. If
    not specified, the session name is used. The extension `.ecg` is appended if not already specified.
    \item \inltt{OutputFrequency} specifies the number of resolution of the
    electrogram data.
    \item \inltt{Points} specifies a list of coordinates at which electrograms
    are desired. \emph{They must not lie within the domain.}
    \end{itemize}
\end{itemize}


\subsection{Stimuli}
Electrophysiological propagaion is initiated through the stimulus current 
$I_{\mathrm{ion}}$. The \inltt{STIMULI} section describes one or more regions of
stimulus and the time-dependent protocol with which they are applied.
\begin{lstlisting}[style=XmlStyle]
<STIMULI>
  ...
</STIMULI>
\end{lstlisting}
A number of stimulus types are available

\subsubsection{Stimulus types}
\begin{itemize}
    \item \inltt{StimulusRect} stimulates a cuboid-shaped region of the domain,
    specified by two coordinates $(x_1,y_1,z_1)$ and $(x_2,y_2,z_2)$.
    An additional parameter specifies the "smoothness" of the boundaries of the
    region; higher values produce a sharper boundary. Finally, the maximum 
    strength of the stimulus current is specified in $\mu \mathrm{A} / \mathrm{mm}^3$
\begin{lstlisting}[style=XmlStyle]
<STIMULUS TYPE="StimulusRect" ID="0">
  <p_x1> -15.24 </p_x1>
  <p_y1>  14.02 </p_y1>
  <p_z1>   6.87 </p_z1>
  <p_x2>  12.23 </p_x2>
  <p_y2>  16.56 </p_y2>
  <p_z2>   8.88 </p_z2>
  <p_is> 100.00 </p_is>
  <p_strength> 50.0 </p_strength>
</STIMULUS>
\end{lstlisting}

    \item \inltt{StimulusCirc} stimulates a spherical region of the domain, as
    specified by a centre and radius. The smoothness and strength parameters are also specified as for `StimulusRect`.
\begin{lstlisting}[style=XmlStyle]
<STIMULUS TYPE="StimulusCirc" ID="0">
  <p_x1> -15.24 </p_x1>
  <p_y1>  14.02 </p_y1>
  <p_z1>   6.87 </p_z1>
  <p_r1>  12.23 </p_r1>
  <p_is> 100.00 </p_is>
  <p_strength> 50.0 </p_strength>
</STIMULUS>
\end{lstlisting}
\end{itemize}


\subsubsection{Protocols}
A protocol specifies the time-dependent function indicating the strength of the
stimulus and one such \inltt{PROTOCOL} section should be included within each
\inltt{STIMULUS}. This can be expressed as one of:
\begin{itemize}
    \item \inltt{ProtocolSingle} a single stimulus is applied at a given start
    time and for a given duration
\begin{lstlisting}[style=XmlStyle]
<PROTOCOL TYPE="ProtocolSingle">
  <START>    0.0   </START>
  <DURATION> 2.0   </DURATION>
</PROTOCOL>
\end{lstlisting}

    \item \inltt{ProtocolS1} a train of pulses of fixed duration applied at a
    given start time and with a given cycle length.
\begin{lstlisting}[style=XmlStyle]
<PROTOCOL TYPE="ProtocolS1">
  <START>    0.0   </START>
  <DURATION> 2.0   </DURATION>
  <S1CYCLELENGTH> 300.0 </S1CYCLELENGTH>
  <NUM_S1>   5     </NUM_S1>
</PROTOCOL>
\end{lstlisting}

    \item \inltt{ProtocolS1S2} same as `ProtocolS1` except with an additional
    single pulse applied at a different cycle length at the end of the train of S1 pulses.
\begin{lstlisting}[style=XmlStyle]
<PROTOCOL TYPE="ProtocolS1S2">
  <START>    0.0   </START>
  <DURATION> 2.0   </DURATION>
  <S1CYCLELENGTH> 300.0 </S1CYCLELENGTH>
  <NUM_S1>   5     </NUM_S1>
  <S2CYCLELENGTH> 100.0 </S2CYCLELENGTH>
</PROTOCOL>
\end{lstlisting}
\end{itemize}
