%%% DOCUMENTCLASS 
%%%-------------------------------------------------------------------------------

\documentclass[
a4paper, % Stock and paper size.
11pt, % Type size.
% article,
% oneside, 
onecolumn, % Only one column of text on a page.
% openright, % Each chapter will start on a recto page.
% openleft, % Each chapter will start on a verso page.
openany, % A chapter may start on either a recto or verso page.
]{memoir}

%%% PACKAGES 
%%%------------------------------------------------------------------------------

\usepackage[utf8]{inputenc} % If utf8 encoding
% \usepackage[lantin1]{inputenc} % If not utf8 encoding, then this is probably the way to go
\usepackage[T1]{fontenc}    %
\usepackage{lmodern}
\usepackage[english]{babel} % English please
\usepackage[final]{microtype} % Less badboxes

% \usepackage{kpfonts} %Font

\usepackage{amsmath,amssymb,mathtools} % Math

% \usepackage{tikz} % Figures
\usepackage{graphicx} % Include figures


%%% PAGE LAYOUT 
%%%-----------------------------------------------------------------------------
\setlrmarginsandblock{0.15\paperwidth}{*}{1} % Left and right margin
\setulmarginsandblock{0.2\paperwidth}{*}{1}  % Upper and lower margin
\checkandfixthelayout

\newlength\forceindent
\setlength{\forceindent}{\parindent}
\setlength{\parindent}{0cm}
\renewcommand{\indent}{\hspace*{\forceindent}}
\setlength{\parskip}{1em}


%%% SECTIONAL DIVISIONS
%%%------------------------------------------------------------------------------

\maxsecnumdepth{subsection} % Subsections (and higher) are numbered
\setsecnumdepth{subsection}

\makeatletter %
\makechapterstyle{standard}{
  \setlength{\beforechapskip}{0\baselineskip}
  \setlength{\midchapskip}{1\baselineskip}
  \setlength{\afterchapskip}{8\baselineskip}
  \renewcommand{\chapterheadstart}{\vspace*{\beforechapskip}}
  \renewcommand{\chapnamefont}{\centering\normalfont\Large}
  \renewcommand{\printchaptername}{\chapnamefont \@chapapp}
  \renewcommand{\chapternamenum}{\space}
  \renewcommand{\chapnumfont}{\normalfont\Large}
  \renewcommand{\printchapternum}{\chapnumfont \thechapter}
  \renewcommand{\afterchapternum}{\par\nobreak\vskip \midchapskip}
  \renewcommand{\printchapternonum}{\vspace*{\midchapskip}\vspace*{5mm}}
  \renewcommand{\chaptitlefont}{\centering\bfseries\LARGE}
%  \renewcommand{\printchaptertitle}[1]{\chaptitlefont ##1}
  \renewcommand{\afterchaptertitle}{\par\nobreak\vskip \afterchapskip}
}
\makeatother

%\chapterstyle{standard}
\chapterstyle{madsen}

\setsecheadstyle{\normalfont\large\bfseries}
\setsubsecheadstyle{\normalfont\normalsize\bfseries}
\setparaheadstyle{\normalfont\normalsize\bfseries}
\setparaindent{0pt}\setafterparaskip{0pt}

%%% FLOATS AND CAPTIONS
%%%------------------------------------------------------------------------------

\makeatletter                  % You do not need to write [htpb] all the time
\renewcommand\fps@figure{htbp} %
\renewcommand\fps@table{htbp}  %
\makeatother                   %

\captiondelim{\space } % A space between caption name and text
\captionnamefont{\small\bfseries} % Font of the caption name
\captiontitlefont{\small\normalfont} % Font of the caption text

\changecaptionwidth          % Change the width of the caption
\captionwidth{1\textwidth} %

%%% ABSTRACT
%%%------------------------------------------------------------------------------

\renewcommand{\abstractnamefont}{\normalfont\small\bfseries} % Font of abstract title
\setlength{\absleftindent}{0.1\textwidth} % Width of abstract
\setlength{\absrightindent}{\absleftindent}

%%% HEADER AND FOOTER 
%%%------------------------------------------------------------------------------

\makepagestyle{standard} % Make standard pagestyle

\makeatletter                 % Define standard pagestyle
\makeevenfoot{standard}{}{}{} %
\makeoddfoot{standard}{}{}{}  %
\makeevenhead{standard}{\bfseries\thepage\normalfont\qquad\small\leftmark}{}{}
\makeoddhead{standard}{}{}{\small\rightmark\qquad\bfseries\thepage}
% \makeheadrule{standard}{\textwidth}{\normalrulethickness}
\makeatother                  %

\makeatletter
\makepsmarks{standard}{
\createmark{chapter}{both}{shownumber}{\@chapapp\ }{ \quad }
\createmark{section}{right}{shownumber}{}{ \quad }
\createplainmark{toc}{both}{\contentsname}
\createplainmark{lof}{both}{\listfigurename}
\createplainmark{lot}{both}{\listtablename}
\createplainmark{bib}{both}{\bibname}
\createplainmark{index}{both}{\indexname}
\createplainmark{glossary}{both}{\glossaryname}
}
\makeatother                               %

\makepagestyle{chap} % Make new chapter pagestyle

\makeatletter
\makeevenfoot{chap}{}{\small\bfseries\thepage}{} % Define new chapter pagestyle
\makeoddfoot{chap}{}{\small\bfseries\thepage}{}  %
\makeevenhead{chap}{}{}{}   %
\makeoddhead{chap}{}{}{}    %
% \makeheadrule{chap}{\textwidth}{\normalrulethickness}
\makeatother

\nouppercaseheads
\pagestyle{standard}               % Choosing pagestyle and chapter pagestyle
\aliaspagestyle{chapter}{chap} %


%%% NEW COMMANDS
%%%-----------------------------------------------------------------------------


\newcommand{\p}{\partial} %Partial
% Or what ever you want


%%% CODE SNIPPETS, COMMANDS, ETC
%%%-----------------------------------------------------------------------------
\usepackage{xcolor}
\usepackage{listings} % Display code / shell commands
%\newcommand{\shellcommand}[1]{\begin{lstlisting} \#1 \end{lstlisting}
\lstdefinestyle{BashInputStyle}{
  language=bash,
  basicstyle=\small\sffamily,
%  numbers=left,
%  numberstyle=\tiny,
%  numbersep=3pt,
  frame=,
  columns=fullflexible,
  backgroundcolor=\color{black!05},
  linewidth=0.9\linewidth,
  xleftmargin=0.1\linewidth
}
\definecolor{gray}{rgb}{0.4,0.4,0.4}
\definecolor{darkblue}{rgb}{0.0,0.0,0.6}
\definecolor{cyan}{rgb}{0.0,0.6,0.6}
\definecolor{maroon}{rgb}{0.5,0.0,0.0}
\lstdefinelanguage{XML}
{
  basicstyle=\ttfamily\footnotesize,
  morestring=[b]",
  moredelim=[s][\bfseries\color{maroon}]{<}{\ },
  moredelim=[s][\bfseries\color{maroon}]{</}{>},
  moredelim=[l][\bfseries\color{maroon}]{/>},
  moredelim=[l][\bfseries\color{maroon}]{>},
  morecomment=[s]{<?}{?>},
  morecomment=[s]{<!--}{-->},
  commentstyle=\color{gray},
  stringstyle=\color{orange},
  identifierstyle=\color{darkblue}
}
\lstdefinestyle{XMLStyle}{
  language=XML,
  basicstyle=\sffamily\footnotesize,
  numbers=left,
  numberstyle=\tiny,
  numbersep=3pt,
  frame=,
  columns=fullflexible,
  backgroundcolor=\color{black!05},
  linewidth=0.9\linewidth,
  xleftmargin=0.1\linewidth
}


%%% TABLE OF CONTENTS
%%%-----------------------------------------------------------------------------

\maxtocdepth{subsection} % Only parts, chapters and sections in the table of contents
\settocdepth{subsection}

%\AtEndDocument{\addtocontents{toc}{\par}} % Add a \par to the end of the TOC

%%% INTERNAL HYPERLINKS
%%%-----------------------------------------------------------------------------
\usepackage{hyperref}   % Internal hyperlinks
\hypersetup{
pdfborder={0 0 0},      % No borders around internal hyperlinks
pdfauthor={I am the Author} % author
}
\usepackage{memhfixc}   %


%%% PRETTY TITLE PAGE FOR PDF DOC
%%%-----------------------------------------------------------------------------
\makeatletter
\newlength\drop
\newcommand{\br}{\hfill\break}
\newcommand*{\titlepage}{%
    \thispagestyle{empty}
    \begingroup% Gentle Madness
    \drop = 0.1\textheight
    \vspace*{\baselineskip}
    \vfill
    \hbox{%
      \hspace*{0.1\textwidth}%
      \rule{1pt}{\dimexpr\textheight-28pt\relax}%
      \hspace*{0.05\textwidth}% 
      \parbox[b]{0.85\textwidth}{%
        \vbox{%
          \vspace{\drop}
          {\Huge\bfseries\raggedright\@title\par}
          \vskip2.37\baselineskip
          {\huge\bfseries Version 4.0\par}
          \vskip4\baselineskip
          {\huge\bfseries User Guide\par}
          \vskip1.0\baselineskip
          {\large\bfseries\@date\par}
          \vspace{0.3\textheight}
          {\small\noindent\@author}\\[\baselineskip]
        }% end of vbox
      }% end of parbox
    }% end of hbox
    \vfill
    \null
\endgroup}
\makeatother

%%% THE DOCUMENT
%%% Where all the important stuff is included!
%%%-------------------------------------------------------------------------------

\author{Department of Aeronautics, Imperial College London, UK\\
Scientific Computing and Imaging Institute, University of Utah, USA}
\title{Nektar++: Spectral/hp Element Framework}
\date{\today}

\begin{document}

\frontmatter

% Render pretty title page if not building HTML
\ifdefined\HCode
\maketitle
\else
\titlepage
\fi

\clearpage

\ifx\HCode\undefined
\tableofcontents*
\fi

\clearpage

\chapter{Introduction}


\mainmatter

\chapter{Installation}

\section{Installing on Debian/Ubuntu Systems}

\section{Installing of Redhat/Fedora Systems}

\section{Installing from Source}


\chapter{Solvers}


\chapter{XML Input File Reference}
\label{s:xml}

The Nektar++ native file format is compliant with XML version 1.0. The root element is NEKTAR and has the overall structure as follows
\begin{lstlisting}[style=XMLStyle]
<NEKTAR>
  <GEOMETRY>
  ...
  </GEOMETRY>
  <EXPANSIONS>
  ...
  </EXPANSIONS>
  <CONDITIONS>
  ...
  </CONDITIONS>
  <FILTERS>
  ...
  </FILTERS>
  <GLOBALOPTIMIZATIONPARAMETERS>
  ...
  </GLOBALOPTIMIZATIONPARAMETERS>
</NEKTAR>
\end{lstlisting}

\section{Geometry}
This section defines the mesh. It specifies a list of vertices, edges (in two or three dimensions) and faces (in three dimensions) and how they connect to create the elemental decomposition of the domain. It also defines a list of composites which are used in the Expansions and Conditions sections of the file to describe the polynomial expansions and impose boundary conditions.

The GEOMETRY section is structured as
\begin{lstlisting}[style=XMLStyle]
<GEOMETRY DIM="2" SPACE="2">
  <VERTEX>
  ...
  </VERTEX>
  <EDGE>
  ...
  </EDGE>
  <FACE>
  ...
  </FACE>
  <ELEMENT>
  ...
  </ELEMENT>
  <CURVED>
  ...
  </CURVED>
  <COMPOSITE>
  ...
  </COMPOSITE>
  <DOMAIN> ... </DOMAIN>
</GEOMETRY>
\end{lstlisting}
It has two attributes:

- DIM: specifies the dimension of the expansion elements.
- SPACE: specifies the dimension of the space in which the elements exist.

These attributes allow, for example, a two-dimensional surface to be embedded in a three-dimensional space. The FACES section is only present when DIM=3. The CURVED section is only present if curved edges or faces are present in the mesh.

\subsection{Vertices}

Vertices have three coordinates. Each has a unique vertex ID. They are defined in the file within VERTEX subsection as follows:
\begin{lstlisting}[style=XMLStyle]
<VERTEX>
  <V ID="0"> 0.0  0.0  0.0 </V>
  ...
</VERTEX>
\end{lstlisting}
VERTEX subsection has three optional attributes: {{{XSCALE}}}, {{{YSCALE}}} and {{{ZSCALE}}}. They specify scaling factors to corresponding vertex coordinates. For example, the following snippet
\begin{lstlisting}[style=XMLStyle]
<VERTEX XSCALE="5">
  <V ID="0"> 0.0  0.0  0.0 </V>
  <V ID="1"> 1.0  2.0  0.0 </V>
</VERTEX>
\end{lstlisting}
defines two vertices with coordinates [[formula( (0.0,0.0,0.0), (5.0,2.0,0.0) )]]. Values of {{{XSCALE}}}, {{{YSCALE}}} and {{{ZSCALE}}} attributes can be arbitrary [wiki:Reference/AnalyticExpressions analytic expressions] depending on pre-defined constants, parameters defined earlier in the XML file and mathematical operations/functions of the latter. If omitted, default scaling factors 1.0 are assumed.

\subsection{Edges}

Edges are defined by two vertices. Each edge has a unique edge ID. They are defined in the file with a line of the form
\begin{lstlisting}[style=XMLStyle]
<E ID="0"> 0 1 </E>
\end{lstlisting}

\subsection{Faces}

Faces are defined by three or more edges. Each face has a unique face ID. They are defined in the file with a line of the form
\begin{lstlisting}[style=XMLStyle]
<T ID="0"> 0 1 2 </T>
<Q ID="1"> 3 4 5 6 </Q>
\end{lstlisting}

The choice of tag specified (T or Q), and thus the number of edges specified depends on the geometry of the face (triangle or quadrilateral).

\subsection{Element}

Elements define the top-level geometric entities in the mesh. Their definition depends upon the dimension of the expansion. For two-dimensional expansions, an element is defined by a sequence of three or four edges. For three-dimensional expansions, the element is defined by a list of faces. Elements are defined in the file with a line of the form
\begin{lstlisting}[style=XMLStyle]
<T ID="0"> 0 1 2 </T>
<H ID="1"> 3 4 5 6 7 8 </H>
\end{lstlisting}
Again, the choice of tag specified depends upon the geometry of the element. The element tags are:

- S: Segment
- T: Triangle
- Q: Quadrilateral
- A: Tetrahedron
- P: Pyramid
- R: Prism
- H: Hexahedron

\subsection{Curved Edges and Faces}

For mesh elements with curved edges and/or curved faces, a separate entry is used to describe the control points for the curve. Each curve has a unique curve ID and is associated with a predefined edge or face. The total number of points in the curve (including end points) and their distribution is also included as attributes. The control points are listed in order, each specified by three coordinates. Curved edges are defined in the file with a line of the form
\begin{lstlisting}[style=XMLStyle]
<E ID="3" EDGEID="7" TYPE="PolyEvenlySpaced" NUMPOINTS="3">
    0.0  0.0  0.0    0.5  0.5  0.0    1.0  0.0  0.0
</E>
\end{lstlisting}

\subsection{Composites}

Composites define collections of elements, faces or edges. Each has a unique composite ID associated with it. All components of a composite entry must be of the same type. The syntax allows components to be listed individually or using ranges. Examples include
\begin{lstlisting}[style=XMLStyle]
<C ID="0"> T[0-862] </C>
<C ID="1"> E[68,69,70,71] </C>
\end{lstlisting}

\subsection{Domain}

This tag specifies composites which describe the entire problem domain. It has the form of
\begin{lstlisting}[style=XMLStyle]
<DOMAIN> C[0] </DOMAIN>
\end{lstlisting}


\section{Expansions}
This section defines the polynomial expansions used on each of the defined geometric composites. Expansion entries specify the number of modes, the basis type and have the form
\begin{lstlisting}[style=XMLStyle]
<E COMPOSITE="C[0]" NUMMODES="5" FIELDS="u" TYPE="MODIFIED" />
\end{lstlisting}
or, if we have more then one variable
\begin{lstlisting}[style=XMLStyle]
<E COMPOSITE="C[0]" NUMMODES="5" FIELDS="u,v,p" TYPE="MODIFIED" />
\end{lstlisting}

The expansion basis can also be specified by parts, and thus the user is able to increase the quadrature order. For tet elements this takes the form:

\begin{lstlisting}[style=XMLStyle]
<E COMPOSITE="C[0]" BASISTYPE="Modified_A,Modified_B,Modified_C" NUMMODES="3,3,3" POINTSTYPE="GaussLobattoLegendre,GaussRadauMAlpha1Beta0,GaussRadauMAlpha2Beta0" NUMPOINTS="4,3,3"  FIELDS="u" />
\end{lstlisting}
and for prism elements:
\begin{lstlisting}[style=XMLStyle]
<E COMPOSITE="C[1]" BASISTYPE="Modified_A,Modified_A,Modified_B" NUMMODES="3,3,3" POINTSTYPE="GaussLobattoLegendre,GaussLobattoLegendre,GaussRadauMAlpha1Beta0" NUMPOINTS="4,4,3"  FIELDS="u" />
\end{lstlisting}


\section{Conditions}
The final section of the file defines parameters and boundary conditions which define the nature of the problem to be solved. These are enclosed in the CONDITIONS tag.

\subsection{Parameters}

Parameters may be required by a particular solver (for instance time-integration parameters or solver-specific parameters), or arbitrary and only used within the context of the session file (e.g. parameters in the definition of an initial condition). All parameters are enclosed in the PARAMETERS XML element.
\begin{lstlisting}[style=XMLStyle]
<PARAMETERS>
...
</PARAMETERS>
\end{lstlisting}
A parameter may be of integer or real type and may reference other parameters defined previous to it. It is expressed in the file as

\begin{lstlisting}[style=XMLStyle]
<P> [PARAMETER NAME] = [PARAMETER VALUE] </P>
\end{lstlisting}
For example,

\begin{lstlisting}[style=XMLStyle]
<P> NumSteps = 1000              </P>
<P> TimeStep = 0.01              </P>
<P> FinTime  = NumSteps*TimeStep </P>
\end{lstlisting}

\subsection{Solver Information}

These specify properties to define the actions specific to solvers, typically including the equation to solve, the projection type and the method of time integration. The property/value pairs are specified as XML attributes. For example,
\begin{lstlisting}[style=XMLStyle]
<SOLVERINFO>
  <I PROPERTY="EQTYPE"                VALUE="UnsteadyAdvection"    />
  <I PROPERTY="Projection"            VALUE="Continuous"           />
  <I PROPERTY="TimeIntegrationMethod" VALUE="ClassicalRungeKutta4" />
</SOLVERINFO>
\end{lstlisting}

The list of available solvers in Nektar++ can be found [wiki:Tutorial here].


\subsection{Variables}

These define the number (and name) of solution variables. Each variable is prescribed a unique ID. For example a two-dimensional flow simulation may define the velocity variables  using
\begin{lstlisting}[style=XMLStyle]
<VARIABLES>
  <V ID="0"> u </V>
  <V ID="1"> v </V>
</VARIABLES>
\end{lstlisting}

\subsection{Global System Solution Information}

This section allows you to specify the global system solution parameters which is particularly useful when using an iterative solver. An example of this section is as follows: 

\begin{lstlisting}[style=XMLStyle]
<GLOBALSYSSOLNINFO>
  <V VAR="u,v,w">
    <I PROPERTY="GlobalSysSoln"    VALUE="IterativeStaticCond" />
    <I PROPERTY="Preconditioner"   VALUE="LowEnergyBlock"/>
    <I PROPERTY="IterativeSolverTolerance"  VALUE="1e-8"/>
  </V>
  <V VAR="p">
    <I PROPERTY="GlobalSysSoln"    VALUE="IterativeStaticCond" />
    <I PROPERTY="Preconditioner"   VALUE="FullLinearSpaceWithLowEnergyBlock"/>
    <I PROPERTY="IterativeSolverTolerance"  VALUE="1e-6"/>
  </V>
</GLOBALSYSSOLNINFO>
\end{lstlisting}

The above section specifies that the global solution system for the variables "u,v,w" should use the iIerativeStaticCond approach with the LowEnergyBlock preconditioned and an iterative tolerance of 1e-6.  Where as the variable "p" which also si sovlerd with the IterativeStaticCond approach should use the FullLinearSpaceWithLowEnergyBlock and an iterative tolerance of 1e-8. 

Other parameters which can be specified include SuccessiveRHS. 

The parameters in this section override those specified in the Parameters section. 

\subsection{Boundary Regions and Conditions}

Boundary conditions are defined by two XML elements. The first defines the various boundary regions in the domain in terms of composite entities from the GEOMETRY section of the file. Each boundary region has a unique ID and are defined as, for example,
\begin{lstlisting}[style=XMLStyle]
<BOUNDARYREGIONS>
  <B ID="0"> C[2] </B>
  <B ID="1"> C[3] </B>
</BOUNDARYREGIONS>
\end{lstlisting}
The second defines the actual boundary condition to impose on that composite region for each of the defined solution variables, and has the form,
\begin{lstlisting}[style=XMLStyle]
<BOUNDARYCONDITIONS>
  <REGION REF="0">
    <D VAR="u" VALUE="sin(PI*x)*cos(PI*y)" />
    <D VAR="v" VALUE="sin(PI*x)*cos(PI*y)" />
  </REGION>
</BOUNDARYCONDITIONS>
\end{lstlisting}
Boundary condition specifications may refer to any parameters defined in the session file. The REF attribute corresponds to a defined boundary region. The tag used for each variable specifies the type of boundary condition to enforce. These can be either
 - D: Dirichlet
 - N: Neumann
 - R: Robin
 - P: Periodic

[wiki:Reference/BoundaryConditionTypes This page] provides the list of all acceptable boundary condition types and syntax of their declarations.

Time-dependent boundary conditions may be specified through setting the USERDEFINEDTYPE attribute. For example,
\begin{lstlisting}[style=XMLStyle]
<D VAR="u" USERDEFINEDTYPE="TimeDependent" VALUE="sin(PI*(x-t))" />
\end{lstlisting}
Periodic boundary conditions reference the corresponding boundary region with which to enforce periodicity.

The following example provides an example of three boundary conditions for a two-dimensional flow,
\begin{lstlisting}[style=XMLStyle]
<BOUNDARYCONDITIONS>
  <REGION REF="0">
    <D VAR="u" USERDEFINEDTYPE="TimeDependent" VALUE="-cos(x)*sin(y)*exp(-2*t*Kinvis)" />
    <D VAR="v" USERDEFINEDTYPE="TimeDependent" VALUE="sin(x)*cos(y)*exp(-2*t*Kinvis)" />
    <P VAR="p" VALUE=[2]/>
  </REGION>
  <REGION REF="1">
    <D VAR="u" USERDEFINEDTYPE="TimeDependent" VALUE="-cos(x)*sin(y)*exp(-2*t*Kinvis)" />
    <D VAR="v" USERDEFINEDTYPE="TimeDependent" VALUE="sin(x)*cos(y)*exp(-2*t*Kinvis)" />
    <N VAR="p" USERDEFINEDTYPE="H" VALUE="0.0"/>
  </REGION>
  <REGION REF="2">
    <D VAR="u" USERDEFINEDTYPE="TimeDependent" VALUE="-cos(x)*sin(y)*exp(-2*t*Kinvis)" />
    <D VAR="v" USERDEFINEDTYPE="TimeDependent" VALUE="sin(x)*cos(y)*exp(-2*t*Kinvis)" />
    <P VAR="p" VALUE=[0]/>
  </REGION>
  <REGION REF="3">
    <D VAR="u" USERDEFINEDTYPE="TimeDependent" VALUE="-cos(x)*sin(y)*exp(-2*t*Kinvis)" />
    <D VAR="v" USERDEFINEDTYPE="TimeDependent" VALUE="sin(x)*cos(y)*exp(-2*t*Kinvis)" />
    <D VAR="p" USERDEFINEDTYPE="TimeDependent" VALUE="-0.25*(cos(2*x)+cos(2*y))*exp(-4*t*Kinvis)"/>
  </REGION>
</BOUNDARYCONDITIONS>
\end{lstlisting}
where the boundary regions which are periodic are linked via their region identifier (Region 0 and Region 2).

Boundary conditions can also be loaded from file, here an example from the Incompressible Navier-Stokes cases,
\begin{lstlisting}[style=XMLStyle]
<REGION REF="1">
  <D VAR="u" FILE="Test_ChanFlow2D_bcsfromfiles_u_1.bc" />
  <D VAR="v" VALUE="0" />
  <N VAR="p" USERDEFINEDTYPE="H" VALUE="0" />
</REGION>
\end{lstlisting}

\subsection{Functions}

Finally, multi-variable functions such as initial conditions and analytic solutions may be specified for use in, or comparison with, simulations. These may be specified using expressions (<E>) or imported from a file (<F>) using the Nektar++ FLD file format
\begin{lstlisting}[style=XMLStyle]
<FUNCTION NAME="ExactSolution">
  <E VAR="u" VALUE="sin(PI*x-advx*t))*cos(PI*(y-advy*t))" />
</FUNCTION>
<FUNCTION NAME="InitialConditions">
  <F VAR="u" FILE="session.rst" />
</FUNCTION>
\end{lstlisting}
A restart file is a solution file (in other words an .fld renamed as .rst) where the field data is specified. The expansion order used to generate the .rst file must be the same as that for the simulation. The filename must be specified relative to the location of the .xml file.

Other examples of this input features can be the insertion of a forcing term,
\begin{lstlisting}[style=XMLStyle]
<FUNCTION NAME="BodyForce">
  <E VAR="u" VALUE="0" />
  <E VAR="v" VALUE="0" />
</FUNCTION>
<FUNCTION NAME="Forcing">
  <E VAR="u" VALUE="-(Lambda + 2*PI*PI)*sin(PI*x)*sin(PI*y)" />
</FUNCTION>
\end{lstlisting}
or of a linear advection term
\begin{lstlisting}[style=XMLStyle]
<FUNCTION NAME="AdvectionVelocity">
  <E VAR="Vx" VALUE="1.0" />
  <E VAR="Vy" VALUE="1.0" />
  <E VAR="Vz" VALUE="1.0" />
</FUNCTION>
\end{lstlisting}

[wiki:Reference/AnalyticExpressions This page] provides the list of acceptable mathematical functions and other related technical details.

\subsection{Quasi-3D approach}

To generate a Quasi-3D appraoch with Nektar++ we only need to create a 2D or a 1D mesh, as reported above, and then specify the parameters to extend the problem to a 3D case. For a 2D spectral/hp element problem, we have a 2D mesh and along with the parameters we need to define the problem (i.e. equation type, boundary conditions, etc.). The only thing we need to do, to extend it to a Quasi-3D approach, is to specify some additional parameters which characterise the harmonic expansion in the third direction. First we need to specify in the solver information section that that the problem will be extended to have one homogeneouns dimension; here an example
\begin{lstlisting}[style=XMLStyle]
<SOLVERINFO>
  <I PROPERTY="SolverType" VALUE="VelocityCorrectionScheme"/>
  <I PROPERTY="EQTYPE" VALUE="UnsteadyNavierStokes"/>
  <I PROPERTY="AdvectionForm" VALUE="Convective"/>
  <I PROPERTY="Projection" VALUE="Galerkin"/>
  <I PROPERTY="TimeIntegrationMethod" VALUE="IMEXOrder2"/>
  <I PROPERTY="HOMOGENEOUS" VALUE="1D"/>
</SOLVERINFO>
\end{lstlisting}
then we need to specify the parameters which define the 1D harmonic expanson along the z-axis, namely the homogeneous length (LZ) and the number of modes in the homogeneous direction (HomModesZ). HomModesZ corresponds also to the number of quadrature points in the homogenous direction, hence on the number of 2D planes discretized with a specral/hp element method.
\begin{lstlisting}[style=XMLStyle]
<PARAMETERS>
  <P> TimeStep      = 0.001   </P>
  <P> NumSteps      = 1000    </P>
  <P> IO_CheckSteps = 100     </P>
  <P> IO_InfoSteps  = 10      </P>
  <P> Kinvis        = 0.025   </P>
  <P> HomModesZ     = 4       </P>
  <P> LZ            = 1.0     </P>
</PARAMETERS>
\end{lstlisting}
In case we want to create a Quasi-3D approach starting form a 1D spectral/hp element mesh, the procedure is the same, but we need to specify the parameters for two harmonic directions (in Y and Z direction). For Example,
\begin{lstlisting}[style=XMLStyle]
<SOLVERINFO>
  <I PROPERTY="EQTYPE" VALUE="UnsteadyAdvectionDiffusion" />
  <I PROPERTY="Projection" VALUE="Continuous"/>
  <I PROPERTY="HOMOGENEOUS" VALUE="2D"/>
  <I PROPERTY="DiffusionAdvancement" VALUE="Implicit"/>
  <I PROPERTY="AdvectionAdvancement" VALUE="Explicit"/>
  <I PROPERTY="TimeIntegrationMethod" VALUE="IMEXOrder2"/>
</SOLVERINFO>
<PARAMETERS>
  <P> TimeStep      = 0.001 </P>
  <P> NumSteps      = 200   </P>
  <P> IO_CheckSteps = 200   </P>
  <P> IO_InfoSteps  = 10    </P>
  <P> wavefreq      = PI    </P>
  <P> epsilon       = 1.0   </P>
  <P> Lambda        = 1.0   </P>
  <P> HomModesY     = 10    </P>
  <P> LY            = 6.5   </P>
  <P> HomModesZ     = 6     </P>
  <P> LZ            = 2.0   </P>
</PARAMETERS>
\end{lstlisting}
By default the opeartions associated with the harmonic expansions are performed with the Matix-Vector-Multiplication (MVM) defined inside the code. The Fast Fourier Transofrm (FFT) can be used to speed up the operations (if the FFTW library has been compiled in ThirdParty, see the compilation instructions). To use the FFT routines we need just to insert a flag in the solver information as below:
\begin{lstlisting}[style=XMLStyle]
<SOLVERINFO>
  <I PROPERTY="EQTYPE" VALUE="UnsteadyAdvectionDiffusion" />
  <I PROPERTY="Projection" VALUE="Continuous"/>
  <I PROPERTY="HOMOGENEOUS" VALUE="2D"/>
  <I PROPERTY="DiffusionAdvancement" VALUE="Implicit"/>
  <I PROPERTY="AdvectionAdvancement" VALUE="Explicit"/>
  <I PROPERTY="TimeIntegrationMethod" VALUE="IMEXOrder2"/>
  <I PROPERTY="USEFFT" VALUE="FFTW"/>
</SOLVERINFO>
\end{lstlisting}
The number of homogenenous modes has to be even. The Quasi-3D apporach can be created starting from a 2D mesh and adding one homogenous expansion or starting form a 1D mesh and adding two homogeneous expansions. Not other options available. In case of a 1D homogeneous extension, the homogeneous direction will be the z-axis. In case of a 2D homogeneous extension, the homogeneous directions will be the y-axis and the z-axis.



\section{Analytic Expressions}
This section discusses particulars related to analytic expressions appearing in
Nektar++. Analytic expressions in Nektar++ are used to describe spatially or
temporally varying properties, for example
\begin{itemize}
\item velocity profiles on a boundary
\item some reference functions (e.g. exact solutions)
\end{itemize}
which can be retrieved in the solver code.

Analytic expressions appear as the content of \inltt{VALUE} attribute of
\begin{itemize}
\item boundary condition type tags within \inltt{<REGION>} subsection of
 \inltt{<BOUNDARYCONDITIONS>}, e.g. \inltt{<D>}, \inltt{<N>} etc. 
 %See [wiki:BoundaryConditionTypes] for details.
\item expression declaration tag \inltt{<E>} within \inltt{<FUNCTION>}
subsection.
\end{itemize}

The tags above declare analytic expressions as well as link them to one of the
field variables declared in \inltt{<EXPANSIONS>} section. For example, the
declaration 
\begin{lstlisting}[style=XMLStyle]
  <D VAR="u" VALUE="sin(PI*x)*cos(PI*y)" />
\end{lstlisting}
registers expression $\sin(\pi x)\cos(\pi y)$ as a Dirichlet
boundary constraint associated with field variable \inltt{u}.

Enforcing the same velocity profile at multiple boundary regions and/or field
variables results in repeated re-declarations of a corresponding analytic
expression. Currently one cannot directly link a boundary condition declaration
with an analytic expression uniquely specified somewhere else, e.g. in the
\inltt{<FUNCTION>} subsection. However this duplication does not affect an
overall computational performance.

% \subsection{Ordering of tags}
% 
% TODO Here one should describe the constraints of internal !SessionReader API for
% expression retrieval via function name and its ordering number. Ordering of tags
% is important for the user code. Not everything has a name for the solver code.

\subsection{Variables and coordinate systems}
Declarations of analytic expressions are formulated in terms of problem
space-time coordinates. The library code makes a number of assumptions to
variable names and their order of appearance in the declarations. This section
describes these assumptions.

Internally, the library uses 3D global coordinate space regardless of problem
dimension. Internal global coordinate system has natural basis
{{{(1,0,0),(0,1,0),(0,0,1)}}} with coordinates '''x''','''y''' and '''z'''. In
other words, variables '''x''','''y''' and '''z''' are considered to be first,
second and third coordinates of a point ('''x''','''y''','''z''').

Declarations of problem spatial variables do not exist in the current XML file
format. Even though field variables are declarable as in the following code
snippet, 
\begin{lstlisting}[style=XMLStyle]
   <VARIABLES>
     <V ID="0"> u </V>
     <V ID="1"> v </V>
   </VARIABLES>
\end{lstlisting} 
there are no analogous tags for space variables. However an attribute
\inlsh{SPACE} of \inlsh{<GEOMETRY>} section tag declares the dimension of
problem space. For example, \begin{lstlisting}[style=XMLStyle]
  <GEOMETRY DIM="1" SPACE="2"> ...
  </GEOMETRY>
\end{lstlisting}
specifies 1D flow within 2D problem space. The number of spacial variables
presented in expression declaration should match space dimension declared via
\inltt{<GEOMETRY>} section tag.

The library assumes the problem space also has natural basis and spatial
coordinates have names '''x''','''y''' and'''z'''.

Problem space is naturally embedded into the global coordinate space: each point
of
\begin{itemize}
\item 1D problem space with coordinate {{{x}}} is represented by 3D point
 {{{(x,0,0)}}} in the global coordinate system;
\item 2D problem space with coordinates {{{(x,y)}}} is represented by 3D point 
 {{{(x,y,0)}}} in the global coordinate system;
\item 3D problem space with coordinates {{{(x,y,z)}}} has the
 same coordinates in the global space coordinates.
\end{itemize}

Currently, there is no way to describe rotations and translations of problem
space relative to the global coordinate system.

The list of variables allowed in analytic expressions depends on the problem
dimension:
\begin{itemize}
\item For 1D problem analytic expressions must make use of variable '''x'''
only;
\item For 2D problem analytic expressions should make use of variables '''x'''
and '''y'''.
\item 3D problems may use variables '''x''', '''y''' and '''z''' in their
analytic expressions.
\end{itemize}

Violation of these constraints yields unpredictable results of expression
evaluation. The current implementation assigns magic value -9999 to each
dimensionally excessive spacial variable appearing in analytic expressions. For
example, the following declaration 
\begin{lstlisting}[style=XMLStyle]
  <GEOMETRY DIM="2" SPACE="2"> ...
  </GEOMETRY> ...
  <CONDITIONS> ...
    <BOUNDARYCONDITIONS>
       <REGION REF="0">
         <D VAR="u" VALUE="x+y+z" /> <D VAR="v" VALUE="sin(PI*x)*cos(PI*y)" />
       </REGION>
     </BOUNDARYCONDITIONS>
  ...
  </CONDITIONS>
\end{lstlisting}
results in expression $x+y+z$ being evaluated at spatial points
$(x_i,y_i, -9999)$ where $x_i$ and $y_i$ are
the spacial coordinates of boundary degrees of freedom. However, the library
behaviour under this constraint violation may change at later stages of
development (e.g., magic constant 0 may be chosen) and should be considered
unpredictable.

Another example of unpredictable behaviour corresponds to wrong ordering of
variables:
\begin{lstlisting}[style=XMLStyle]
  <GEOMETRY DIM="1" SPACE="1"> ...
  </GEOMETRY> ...
  <CONDITIONS> ...
    <BOUNDARYCONDITIONS>
       <REGION REF="0">
         <D VAR="u" VALUE="sin(y)" />
       </REGION>
     </BOUNDARYCONDITIONS>
  ...
  </CONDITIONS>
\end{lstlisting}
Here one declares 1D problem, so Nektar++ library assumes spacial variable
is '''x'''. At the same time, an expression $sin(y)$ is perfectly
valid on its own, but since it does not depend on '''x''', it will be evaluated
to constant $sin(-9999)$ regardless of degree of freedom under
consideration.

\subsubsection{Time dependence}

Variable '''t''' represents time dependence within analytic expressions. The
boundary condition declarations need to add an additional property
\inltt{USERDEFINEDTYPE="TimeDependent"} in order to flag time dependency to
the library.

% TODO:
%  * check there are no cases when the library evaluates analytic expressions with
%  non-zero time values even though {{{TimeDependent}}} property is not defined *
%  discuss time dependence of functions declared within {} section

\subsubsection{Syntax of analytic expressions}

Analytic expressions are formed of
\begin{itemize}
\item brackets {{{()}}}. Bracketing structure must be balanced.
\item real numbers: every representation is allowed that is correct for
\inlsh{boost::lexical\_cast<double>()}, e.g.
\begin{lstlisting}[style=XMLStyle]
   1.2, 1.2e-5, .02
\end{lstlisting}
\item mathematical constants
\begin{center}
\begin{tabular}{lcc}
\toprule
Identifier & Meaning & Real Value \\
\midrule
\multicolumn{3}{c}{\textbf{Fundamental constants}} \\
E           & Natural Logarithm     & 2.71828182845904523536 \\
PI          & $\pi$                 & 3.14159265358979323846 \\
GAMMA       & Euler Gamma           & 0.57721566490153286060 \\
DEG         & deg/radian            & 57.2957795130823208768 \\
PHI         & golden ratio          & 1.61803398874989484820 \\
\multicolumn{3}{c}{\textbf{Derived constants}} \\
LOG2E       & $\log_2 e$            & 1.44269504088896340740 \\
LOG10E      & $\log_{10} e$         & 0.43429448190325182765 \\
LN2         & $\log_e 2$            & 0.69314718055994530942 \\
PI\_2       & $\frac{\pi}{2}$       & 1.57079632679489661923 \\
PI\_4       & $\frac{\pi}{4}$       & 0.78539816339744830962 \\
1\_PI       & $\frac{1}{\pi}$       & 0.31830988618379067154 \\
2\_PI       & $\frac{2}{\pi}$       & 0.63661977236758134308 \\
2\_SQRTPI   & $\frac{2}{\sqrt{\pi}}$& 1.12837916709551257390 \\
SQRT2       & $\sqrt{2}$            & 1.41421356237309504880 \\
SQRT1\_2    & $\frac{1}{\sqrt{2}}$  & 0.70710678118654752440 \\
\bottomrule
\end{tabular}
\end{center}

\item parameters: alphanumeric names with underscores, e.g. \inltt{GAMMA\_123,
GaM123\_45a\_, \_gamma123} are perfectly acceptable parameter names. However
parameter name cannot start with a numeral. Parameters must be defined with
\inltt{<PARAMETERS>...</PARAMETERS>}. Parameters play the role of constants
that may change their values in between of expression evaluations.

\item variables (i.e., \inlsh{x, y, z} and \inlsh{t})
\item unary minus operator (e.g. \inltt{-x}
\item binary arithmetic operators \inltt{+, -, *, /, \^}
   Powering operator allows using real exponents (it is implemented with
   \inlsh{std::pow()} function)
\item boolean comparison operations \inlsh{<, <=, >, >=, ==} evaluate their
sub-expressions to real values 0.0 or 1.0.
\item mathematical functions of one or two arguments:
\begin{center}
\begin{tabular}{ll}
  \toprule
  \textbf{Identifier} & \textbf{Meaning} \\
  \midrule
  \texttt{abs(x)}     & absolute value $|x|$ \\
  \texttt{asin(x)}    & inverse sine $\arcsin x$ \\
  \texttt{acos(x)}    & inverse cosine $\arccos x$ \\
  \texttt{ang(x,y)}   & computes polar coordinate $\theta=\arctan(y/x)$ from $(x,y)$\\
  \texttt{atan(x)}    & inverse tangent $\arctan x$ \\
  \texttt{atan2(y,x)} & inverse tangent function (used in polar transformations) \\
  \texttt{ceil(x)}    & round up to nearest integer $\lceil x\rceil$ \\
  \texttt{cos(x)}     & cosine $\cos x$ \\
  \texttt{cosh(x)}    & hyperbolic cosine $\cosh x$ \\
  \texttt{exp(x)}     & exponential $e^x$ \\
  \texttt{fabs(x)}    & absolute value (equivalent to \texttt{abs}) \\
  \texttt{floor(x)}   & rounding down $\lfloor x\rfloor$ \\
  \texttt{log(x)}     & logarithm base $e$, $\ln x = \log x$ \\
  \texttt{log10(x)}   & logarithm base 10, $\log_{10} x$ \\
  \texttt{rad(x,y)}   & computes polar coordinate $r=\sqrt{x^2+y^2}$ from $(x,y)$\\
  \texttt{sin(x)}     & sine $\sin x$ \\
  \texttt{sinh(x)}    & hyperbolic sine $\sinh x$ \\
  \texttt{sqrt(x)}    & square root $\sqrt{x}$ \\
  \texttt{tan(x)}     & tangent $\tan x$ \\
  \texttt{tanh(x)}    & hyperbolic tangent $\tanh x$ \\
  \bottomrule
\end{tabular}
\end{center}

These functions are implemented by means of the cmath library:
\url{http://www.cplusplus.com/reference/clibrary/cmath/}. Underlying data type
is \inltt{double} at each stage of expression evaluation. As consequence,
complex-valued expressions (e.g. $(-2)^0.123$) get value \inlsh{nan} (not a
number). The operator \inlsh{\^} is implemented via call to \inlsh{std::pow()}
function and accepts arbitrary real exponents.

\item random noise generation functions. Currently implemented is
\inltt{awgn(sigma)} - Gaussian Noise generator, where $\sigma$ is the variance
of normal distribution with zero mean. Implemented using the
\texttt{boost::mt19937} random number generator with boost variate generators
(see \url{http://www.boost.org/libs/random})
\end{itemize}

\subsection{Performance considerations}
Processing analytic expressions is split into two stages:
\begin{itemize}
\item parsing with pre-evaluation of constant sub-expressions,
\item evaluation to a number.
\end{itemize}
Parsing of analytic expressions with their partial evaluation take place at the
time of setting the run up (reading an XML file). Each analytic expression,
after being pre-processed, is stored internally and quickly retrieved when it
turns to evaluation at given spatial-time point(s). This allows to perform
evaluation of expressions at a large number of spacial points with minimal setup
costs.

\subsubsection{Pre-evaluation details}
Partial evaluation of all constant sub-expressions makes no sense in using
derived constants from table above. This means, either make use of pre-defined
constant \inlsh{LN10\^2} or straightforward expression \inlsh{log10(2)\^2}
results in constant \inlsh{5.3018981104783980105} being stored internally
after pre-processing. The rules of pre-evaluation are as follows:
\begin{itemize}
\item constants, numbers and their combinations with arithmetic, analytic and
 comparison operators are pre-evaluated,
\item appearance of a variable or parameter
 at any recursion level stops pre-evaluation of all upper level operations (but
 doesn't stop pre-evaluation of independent parallel sub-expressions).
\end{itemize}

For example, declaration 
\begin{lstlisting}[style=XMLStyle]
     <D VAR="u" VALUE="exp(-x*sin(PI*(sqrt(2)+sqrt(3))/2)*y )" />
\end{lstlisting}
results in expression \inlsh{exp(-x*(-0.97372300937516503167)*y )} being
stored internally: sub-expression \inlsh{sin(PI*(sqrt(2)+sqrt(3))/2)} is
evaluated to constant but appearance of \inlsh{x} and \inlsh{y} variables
stops further pre-evaluation.

Grouping predefined constants and numbers together helps. Its useful to put
brackets to be sure your constants do not run out and become factors of some
variables or parameters.

Expression evaluator does not do any clever simplifications of input
expressions, which is clear from example above (there is no point in double
negation). The following subsection addresses the simplification strategy.

\subsubsection{Preparing analytic expression}

The total evaluation cost depends on the overall number of operations. Since
evaluator is not making simplifications, it worth trying to minimise the total
number of operations in input expressions manually.

Some operations are more computationally expensive than others. In an order of
increasing complexity:
\begin{itemize}
\item \inlsh{+, -, <, >, <=, >=, ==, }
\item \inlsh{*, /, abs, fabs, ceil, floor,}
\item \inlsh{\^, sqrt, exp, log, log10, sin, cos, tan, sinh, cosh, tanh, asin,
acos, atan}.
\end{itemize}

For example,
\begin{itemize}
\item \inlsh{x*x} is faster than \inlsh{x\^2} --- it is one double
multiplication vs generic calculation of arbitrary power with real exponents.
\item \inlsh{(x+sin(y))\^2} is faster than \inlsh{(x+sin(y))*(x+sin(y))} -
sine is an expensive operation. It is cheaper to square complicated expression rather than
 compute it twice and add one multiplication.
\item An expression
\inltt{exp(-41*( (x+(0.3*cos(2*PI*t)))\^2 + (0.3*sin(2*PI*t))\^2 ))}
 makes use of 5 expensive operations (\inlsh{exp}, \inlsh{sin}, \inlsh{cos}
 and power \inlsh{\^} twice) while an equivalent expression
\inltt{exp(-41*( x*x+0.6*x*cos(2*PI*t) + 0.09 ))}
 uses only 2 expensive operations.
\end{itemize}

If any simplifying identity applies to input expression, it may worth applying
it, provided it minimises the complexity of evaluation. Computer algebra systems
may help.

\subsubsection{Vectorized evaluation}

Expression evaluator is able to calculate an expression for either given point
(its space-time coordinates) or given array of points (arrays of their
space-time coordinates, it uses SoA). Vectorized evaluation is faster then
sequential due to a better data access pattern. Some expressions give measurable
speedup factor $4.6$. Therefore, if you are creating your own solver, it
worth making vectorized calls.
%%% Local Variables: 
%%% mode: latex
%%% TeX-master: "../user-guide"
%%% End: 



\chapter{Command-line Options}

%\begin{lstlisting}
%--verbose
%\end{lstlisting}
\lstinline[style=BashInputStyle]{--verbose}\\
\hangindent=1.5cm
Displays extra info.

\lstinline[style=BashInputStyle]{--help}\\
\hangindent=1.5cm
Displays help information about the available command-line options for the executable.

\lstinline[style=BashInputStyle]{--parameter [key]=[value]}\\
\hangindent=1.5cm
Override a parameter (or define a new one) specified in the XML file.

\lstinline[style=BashInputStyle]{--solverinfo [key]=[value]}\\
\hangindent=1.5cm
Override a solverinfo (or define a new one) specified in the XML file.

\lstinline[style=BashInputStyle]{--shared-filesystem}\\
\hangindent=1.5cm
By default when running in parallel the complete mesh is loaded by all processes, although partitioning is done uniquely on the root process only and communicated to the other processes. Each process then writes out its own partition to the local working directory. This is the most robust approach in accounting for systems where the distributed nodes do not share a common filesystem. In the case that there is a common filesystem, this option forces only the root process to load the complete mesh, perform partitioning and write out the session files for all partitions. This avoids potential memory issues when multiple processes attempt to load the complete mesh on a single node.

\lstinline[style=BashInputStyle]{--npx [int]}\\
\hangindent=1.5cm
When using a fully-Fourier expansion, specifies the number of processes to use in the x-coordinate direction.

\lstinline[style=BashInputStyle]{--npy [int]}\\
\hangindent=1.5cm
\quad When using a fully-Fourier expansion or 3D expansion with two Fourier directions, specifies the number of processes to use in the y-coordinate direction.

\lstinline[style=BashInputStyle]{--npz [int]}\\
\hangindent=1.5cm
When using Fourier expansions, specifies the number of processes to use in the z-coordinate direction.


\chapter{Frequently Asked Questions}


%\appendix


\backmatter

%%% BIBLIOGRAPHY
%%% -------------------------------------------------------------

% \bibliographystyle{utphysics}
% \bibliography{ref}

\end{document}
