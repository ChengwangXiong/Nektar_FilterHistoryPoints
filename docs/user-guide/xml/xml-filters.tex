\section{Filters}

This section defines a range of useful utilities which can be used when specific solvers are run. These are enclosed within the
\inltt{FILTERS} tag. In this section we give an overview of the modules currently available and the fundamental set-up. 

\subsection{Aerodynamic Forces module}

This filter evaluates the aerodynamic forces along a specific surface when the IncompressibleNavierStokes solver. is used The forces are projected along the cartesian axes and the pressure and viscous contributions are computed in each direction. Inside the \inltt{FILTERS} tag we specify the specific filter \inltt{AeroForces} followed by the specific \inltt{PARAMETERS}:

\begin{enumerate}
\item \inltt{OutputFile} specifies the name of the file where the aerodynamic forces are plotted.
\item  \inltt{Frequency} specifies the number of steps to print the aerodynamic forces.
\item \inltt{Boundary} specifies the boundary surfaces where we want to evaluate the aerodynamic forces.
\end{enumerate}

An example is given below:

\begin{lstlisting}[style=XMLStyle]
<FILTERS>
<FILTER TYPE="AeroForces">
<PARAM NAME="OutputFile">DragLift.frc</PARAM>
<PARAM NAME="OutputFrequency">10</PARAM>
<PARAM NAME="Boundary"> B[1,2] </PARAM>		
</PARAM>
</FILTER>
</FILTERS>
\end{lstlisting}

During the execution  a file named DragLift.frc will be created and the value of the aerodynamic forces on boundaries 1 and 2, defined in the \inltt{GEOMETRY} section, will be plotted every 10 time steps .

\subsection{Averaged Fields module}

This filter allows you to compute the averaged fields during time-stepping. Inside the \inltt{FILTERS} tag we specify the specific filter \inltt{AverageFields} followed by the specific \inltt{PARAMETERS}:

\begin{enumerate}
\item \inltt{OutputFile} specifies the name of the file containing the averaged fields.
\item  \inltt{SampleFrequency} specifies the number of instantaneous fields used to evaluate the averaged solution.
\item \inltt{OutputFrequency} specifies the number of steps to print the averaged fields.
\end{enumerate}

\begin{lstlisting}[style=XMLStyle]
<FILTERS>
<FILTER TYPE="AverageFields">
<PARAM NAME="OutputFile">MyAverageField</PARAM>
<PARAM NAME="OutputFrequency">100</PARAM>
<PARAM NAME="SampleFrequency"> 10 </PARAM>		
</PARAM>
</FILTER>
</FILTERS>
\end{lstlisting}

This will create a MyAverageField.fld file averaging the instantaneous fields every 10 time steps. The averaged field is instead plotted every 100 time steps.

\subsection{Check Points module}
 
 The Check Points filter can be used to print the solution during the time integration.  Inside the \inltt{FILTERS} tag we specify \inltt{Checkpoint}, followed by the specific \inltt{PARAMETERS}:

\begin{enumerate}
\item \inltt{OutputFile} specifies the name of the file containing the fields at a specific time step.
\item \inltt{OutputFrequency} specifies the number of steps to print the fields.
\end{enumerate}

For example, to print the fields every 100 time steps we can specify:

\begin{lstlisting}[style=XMLStyle]
<FILTERS>
<FILTER TYPE="Checkpoint">
<PARAM NAME="OutputFile">IntermediateFields</PARAM>
<PARAM NAME="OutputFrequency">100</PARAM>
</PARAM>
</FILTER>
</FILTERS>
\end{lstlisting}

 
\subsection{History Points module}

The history points filters can be used to evaluate the value of the fields in specific points of  the domain during time-stepping.  Inside the \inltt{FILTERS} tag we specify \inltt{HistoryPoints}, followed by the specific \inltt{PARAMETERS}:
 
 \begin{enumerate}
\item \inltt{OutputFile} specifies the name of the file containing the values of the fields in the specific history points.
 \item \inltt{OutputFrequency} specifies the number of steps to print the values of the fields.
 \item \inltt{Points} specifies the coordinates of the specific points line by line.
 \end{enumerate}

An example is provided below:

\begin{lstlisting}[style=XMLStyle]
    <FILTERS>
        <FILTER TYPE="HistoryPoints">
            <PARAM NAME="OutputFile">TimeValues</PARAM>
            <PARAM NAME="OutputFrequency">10</PARAM>
            <PARAM NAME="Points">
                1 0.5 0
                2 0.5 0
                3 0.5 0
            </PARAM>
        </FILTER>
    </FILTERS>
 \end{lstlisting}


\subsection{Modal Energy module}

The modal energy filter can be used to evaluate the kinetic energy of the system and how it distributed along every mode of the Fourier expansion.  Note that this filter is specific to the IncompressibleNavierStokes solver. Inside the \inltt{FILTERS} tag we specify \inltt{ModalEnergy}, followed by the specific \inltt{PARAMETERS}:

 \begin{enumerate}
\item \inltt{OutputFile} specifies the name of the file containing the values of the modal energy.
 \item \inltt{OutputFrequency} specifies the number of steps to print the values of the energy.
 \end{enumerate}

An example is provided below:

\begin{lstlisting}[style=XMLStyle]
    <FILTERS>
        <FILTER TYPE="ModalEnergy">
            <PARAM NAME="OutputFile">EnergyFile</PARAM>
            <PARAM NAME="OutputFrequency">10</PARAM>
            </PARAM>
        </FILTER>
    </FILTERS>
 \end{lstlisting}



This will plot the value of the temporal evolution of the energy and how it is splits on  every mode. 

\subsection {Threshold Value module}

The threshold value filter outputs the time when the solution first exceeds a threshold value.  Inside the \inltt{FILTERS} tag we specify \inltt{ThresholdMax}, followed by the specific \inltt{PARAMETERS}:

 \begin{enumerate}
\item \inltt{ThresholdValue} specifies the threshold value.
 \item \inltt{InitialValue} specifies the initial time.
 \item \inltt{OutputFile} specifies the output file.
 \end{enumerate}
 
 An example is given below:
 
 \begin{lstlisting}[style=XMLStyle]
    <FILTERS>
        <FILTER TYPE="ThresholdMax">
            <PARAM NAME="ThresholdValue "> 0.1 </PARAM>
            <PARAM NAME="InitialValue">  0.4 </PARAM>
             <PARAM NAME="OutputFile">  ThresholdFile </PARAM>
            </PARAM>
        </FILTER>
    </FILTERS>
 \end{lstlisting}
 

