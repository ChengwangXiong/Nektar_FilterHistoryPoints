\section{Expansions}
This section defines the polynomial expansions used on each of the defined
geometric composites. Expansion entries specify the number of modes, the basis
type. The short-hand version has the following form

\begin{lstlisting}[style=XMLStyle]
<E COMPOSITE="C[0]" NUMMODES="5" FIELDS="u" TYPE="MODIFIED" />
\end{lstlisting}

or, if we have more then one variable we can apply the same basis to all using

\begin{lstlisting}[style=XMLStyle]
<E COMPOSITE="C[0]" NUMMODES="5" FIELDS="u,v,p" TYPE="MODIFIED" />
\end{lstlisting}

The expansion basis can also be specified in detail as a combination of
one-dimensional bases, and thus the user is able to, for example, increase the
quadrature order. For tet elements this takes the form:

\begin{lstlisting}[style=XMLStyle]
<E COMPOSITE="C[0]" 
   BASISTYPE="Modified_A,Modified_B,Modified_C" 
   NUMMODES="3,3,3"
   POINTSTYPE="GaussLobattoLegendre,GaussRadauMAlpha1Beta0,GaussRadauMAlpha2Beta0"
   NUMPOINTS="4,3,3"
   FIELDS="u" />
\end{lstlisting}

and for prism elements:

\begin{lstlisting}[style=XMLStyle]
<E COMPOSITE="C[1]" 
   BASISTYPE="Modified_A,Modified_A,Modified_B" 
   NUMMODES="3,3,3"
   POINTSTYPE="GaussLobattoLegendre,GaussLobattoLegendre,GaussRadauMAlpha1Beta0"
   NUMPOINTS="4,4,3"
   FIELDS="u" />
\end{lstlisting}

